%%%%%%%%%%%%%%%%%%%%%%%%%%%%%%%%%%%%%%%%%%%%%%%%
%% Compile the master file!
%% 		Slides: Antonio Machicao y Priemer
%% 		Course: Wissenschaftliches Arbeiten
%%%%%%%%%%%%%%%%%%%%%%%%%%%%%%%%%%%%%%%%%%%%%%%%


%%%%%%%%%%%%%%%%%%%%%%%%%%%%%%%%%%%%%%%%%%%%%%%%%%%%
%%%             Metadata                         
%%%%%%%%%%%%%%%%%%%%%%%%%%%%%%%%%%%%%%%%%%%%%%%%%%%%  

\title{
	\LaTeX\ for Linguists
}

\subtitle{\LaTeX\ 6: Examples \& glossing}

\author[aMyP]{
	{\small Sebastian Nordhoff \& Antonio Machicao y Priemer}
	\\
	{\footnotesize \url{www.linguistik.hu-berlin.de/staff/amyp}}
	%	\\
	%	{\footnotesize \href{mailto:mapriema@hu-berlin.de}{mapriema@hu-berlin.de}}
}

\institute{LOT 2019, Amsterdam}

%\date{ }

%\publishers{\textbf{6. linguistischer Methodenworkshop \\ Humboldt-Universität zu Berlin}}

%\hyphenation{nobreak}


%%%%%%%%%%%%%%%%%%%%%%%%%%%%%%%%%%%%%%%%%%%%%%%%%%%%
%%%             Preamble's End                   
%%%%%%%%%%%%%%%%%%%%%%%%%%%%%%%%%%%%%%%%%%%%%%%%%%%%      


%%%%%%%%%%%%%%%%%%%%%%%%%%%%%%%%%%%
%%%%%%%%%%%%%%%%%%%%%%%%%%%%%%%%%%%    
%% Title slide 
\begin{frame}
  \HUtitle
\end{frame}


%% Contents slide
\frame{
%\begin{multicols}{2}
	\frametitle{Contents}
%	\tableofcontents[hideallsubsections]
	\tableofcontents
	%[pausesections]
%\end{multicols}
	}


%%%%%%%%%%%%%%%%%%%%%%%%%%%%%%%%%%%%
%%%%%%%%%%%%%%%%%%%%%%%%%%%%%%%%%%%%
%% Extra literature

\nocite{Freitag&MyP15a}
\nocite{Knuth1986}
\nocite{Kopka94a}
\nocite{MyP17c}
\nocite{MyP&Kerkhof16a}
	
%%%%%%%%%%%%%%%%%%%%%%%%%%%%%%%%%%%%
%%%%%%%%%%%%%%%%%%%%%%%%%%%%%%%%%%%%


%%%%%%%%%%%%%%%%%%%%%%%%%%%%%%%%%%%%%
%%%%%%%%%%%%%%%%%%%%%%%%%%%%%%%%%%%%%
%%%% Basic literature for these slides
%
%\begin{frame}
%\frametitle{Grundlage \& empfohlene Lektüre}
%
%\dots basierend auf \citet{Freitag&MyP15a} und auf \citet{MyP&Kerkhof16a}\\
%\ras \href{https://www.researchgate.net/publication/279514740_LATEX-Einfuhrung_fur_Linguisten}{LINK}
%
%
%\nocite{Kopka94a}
%
%\end{frame}


%%%%%%%%%%%%%%%%%%%%%%%%%%%%%%%%%%
%%%%%%%%%%%%%%%%%%%%%%%%%%%%%%%%%%
\section{Paket: gb4e}
%\frame{
%	\frametitle{~}
%	\begin{multicols}{2}
%		\tableofcontents[currentsection,hideallsubsections]
%	\end{multicols}
%}
%%%%%%%%%%%%%%%%%%%%%%%%%%%%%%%%%%

\begin{frame}[fragile]
\frametitle{gb4e}

Different packages for examples:

\begin{itemize}
	\item \ltxpack{linguex}
	\item \alert{\ltxpack{gb4e}}
	\item \alert{\ltxpack{langsci-gb4e}}
	\item \dots 
\end{itemize}

\ltxpack{gb4e} has \textbf{more options} than  \ltxpack{linguex}, but it re-defines commands causing problems with the math mode (with \textbf{\_}) and other packages (\fe \ltxpack{forest}).

\bigskip

Load \ltxpack{gb4e} as one of the \textbf{last packages}:

%\begin{itemize}
%	\item \ltxpack{gb4e} definiert bestimmte \LaTeX -Befehle um, und generiert dadurch häufig Probleme mit anderen Paketen. Daher empfiehlt es sich \ltxpack{gb4e} (und auch \ltxpack{hyperref}) als letztes Paket zu laden.
%\end{itemize}

\begin{lstlisting}
\usepackage{gb4e}
\usepackage[hidelinks]{hyperref}
\end{lstlisting}

\end{frame}


%%%%%%%%%%%%%%%%%%%%%%%%%%%%%%%%%%
\begin{frame}[fragile]
%\frametitle{gb4e}

Similar to the \ltxterm{itemize} environment, use an \textbf{\ltxterm{exe} environment} for examples, putting every sentence/phrase in an \lstinline|\ex| \textbf{item}.

%Hier ein Beispiel mit dem \ltxterm{gb4e}-Paket:
{\footnotesize 
\begin{lstlisting}
This is a sample text. 
\begin{exe}
  \ex This is an example.
  \ex This is an example. The only purpose of this text is to show how to work with \LaTeX .
\end{exe}	
This is a sample text. The only purpose of this text is to show how to work 
with \LaTeX .
\end{lstlisting}
}


%\outputbox{
This is a sample text. 
\begin{exe}
\ex This is an example.
\ex This is an example. The only purpose of this text is to show how to work with \LaTeX .
\end{exe}	
This is a sample text. The only purpose of this text is to show how to work with \LaTeX .
%} 
%\vspace{1em}
\end{frame}


%%%%%%%%%%%%%%%%%%%%%%%%%%%%%%%%%%
\begin{frame}[fragile]
%\frametitle{gb4e}

For \textbf{embedded example levels} use the \ltxterm{xlist} \textbf{environment}.

{\footnotesize 
\begin{lstlisting}
\begin{exe}
  \ex This is an example.
  \ex This is an example. The only purpose of this text is to show how to work with \LaTeX .
  \begin{xlist}
    \ex This is an example on a new level.
    \ex This is an example. The only purpose of this text is to show how to work with \LaTeX .
  \end{xlist}		
  \ex This is an example on the first level.
\end{exe}	
\end{lstlisting}
}


%\outputbox{
\begin{exe}
	\ex This is an example.
	\ex This is an example. The only purpose of this text is to show how to work with \LaTeX .
	\begin{xlist}
		\ex This is an example on a new level.
		\ex This is an example. The only purpose of this text is to show how to work with \LaTeX .
	\end{xlist}		
	\ex This is an example on the first level.
\end{exe}	
%} 

\end{frame}


%%%%%%%%%%%%%%%%%%%%%%%%%%%%%%%%%%
\begin{frame}[fragile]
%\frametitle{gb4e}

For \textbf{embedded example levels inside embedded examples} keep using the \ltxterm{xlist} \textbf{environment}.
%Hier ein Beispiel mit dem \ltxterm{gb4e}-Paket:
{\footnotesize 
\begin{lstlisting}
This is a sample text. 
\begin{exe}
  \ex This is an example.
  \ex This is an example. The only purpose of this text is to show how to work with \LaTeX .
  \begin{xlist}
    \ex This is an example on a new level.
    \ex This is an example. The only purpose of this text is to show how to work with \LaTeX .
    \begin{xlist}
      \ex This is an example on a whole new level.
      \ex This is an example. The only purpose of this text is to show how to work with \LaTeX .
    \end{xlist}
    \ex This is an example on the second level.
  \end{xlist}		
  \ex This is an example on the first level.
\end{exe}	
This is a sample text. The only purpose of this text is to show how to work 
with \LaTeX .
\end{lstlisting}
}
\end{frame}

%%%%%%%%%%%%%%%%%%%%%%%%%%%%%%%%%%
\begin{frame}[fragile]
%\frametitle{gb4e}

%\outputbox{
This is a sample text. 
\begin{exe}
	\ex This is an example.
	\ex This is an example. The only purpose of this text is to show how to work with \LaTeX .
	\begin{xlist}
		\ex This is an example on a new level.
		\ex This is an example. The only purpose of this text is to show how to work with \LaTeX .
		\begin{xlist}
			\ex This is an example on a whole new level.
			\ex This is an example. The only purpose of this text is to show how to work with \LaTeX .
		\end{xlist}
		\ex This is an example on the second level.
	\end{xlist}		
	\ex This is an example on the first level.
\end{exe}	
This is a sample text. The only purpose of this text is to show how to work 
with \LaTeX .
%} 
%\vspace{1em}
\end{frame}


%%%%%%%%%%%%%%%%%%%%%%%%%%%%%%%%%%
\begin{frame}[fragile]
%\frametitle{gb4e}

Embedding examples with \textbf{letter numbering} in \textbf{arabic numbering}:

\vspace{-.25cm}

\begin{multicols}{2}

\begin{lstlisting}
\begin{exe}
  \ex %empty!
  \begin{xlist}
    \ex This is an example.
    \ex This is a different example. 
  \end{xlist}		
\end{exe}	
\end{lstlisting}

\columnbreak

%\outputbox{
\begin{exe}
	\ex %empty!
	\begin{xlist}
		\ex This is an example.
		\ex This is a different example. 
	\end{xlist}		
\end{exe}	
%}
\end{multicols}

\textbf{Cross-references} with \ltxterm{label} and \ltxterm{ref}:

\vspace{-.25cm}

\begin{multicols}{2}
	
\begin{lstlisting}
\begin{exe}
  \ex \label{ex:Arabic}%empty!
  \begin{xlist}
    \ex This is an example. 
    \label{ex:Letter1}
    \ex This is a different example. 
    \label{ex:Letter2} 
  \end{xlist}		
\end{exe}	
\end{lstlisting}
	
	\columnbreak

See the following cross-references: (\ref{ex:Arabic}), (\ref{ex:Letter1}), and (\ref{ex:Letter2}).

\vspace{-.1cm}	
%\outputbox{
\begin{exe}
	\ex \label{ex:Arabic}%empty!
	\begin{xlist}
		\ex This is an example. \label{ex:Letter1}
		\ex This is a different example. \label{ex:Letter2} 
	\end{xlist}		
\end{exe}	

%}
\end{multicols}

\end{frame}


%%%%%%%%%%%%%%%%%%%%%%%%%%%%%%%%%%
%%%%%%%%%%%%%%%%%%%%%%%%%%%%%%%%%%
\subsubsection{Acceptability judgements}
%\frame{
%	\frametitle{~}
%	\begin{multicols}{2}
%		\tableofcontents[currentsection,hideallsubsections]
%	\end{multicols}
%}
%%%%%%%%%%%%%%%%%%%%%%%%%%%%%%%%%%
\begin{frame}[fragile]

\frametitle{Acceptability judgements}

For acceptability/grammaticality judgements, use the \textbf{square brackets} and enclose the sentence in \textbf{curly brackets}.

\begin{lstlisting}
\begin{exe}
  \ex[*]{These ungrammatical example is.}
  \ex[]{This example is grammatical.}	
  \ex[\#]{colorless green idea}
  \ex This example is grammatical.
\end{exe}
\end{lstlisting}

%\outputbox{
\begin{exe}
	\ex[*]{These ungrammatical example is.}
	\ex[]{This example is grammatical.}	
	\ex[\#]{colourless green idea}
	\ex This example is grammatical.
\end{exe}
%}

\end{frame}


%%%%%%%%%%%%%%%%%%%%%%%%%%%%%%%%%%
%%%%%%%%%%%%%%%%%%%%%%%%%%%%%%%%%%
\subsubsection{Glossing}
%\frame{
%	\frametitle{~}
%	\begin{multicols}{2}
%		\tableofcontents[currentsection,hideallsubsections]
%	\end{multicols}
%}
%%%%%%%%%%%%%%%%%%%%%%%%%%%%%%%%%%
%%%%%%%%%%%%%%%%%%%%%%%%%%%%%%%%%%
\begin{frame}[fragile]

\frametitle{Glossing}

\begin{enumerate}
	\item The \lstinline|\ex| line remains empty;
	
	\item use \lstinline|\gll| and write in that line your \textbf{example}; \textbf{end} the line it with \lstinline|\\|;
	
	\item write the \textbf{glosses}, and \textbf{end} this line with \lstinline|\\|;
	
	\item \textbf{optionally}, give a \textbf{translation} \lstinline|\glt|.
	
\end{enumerate}


\begin{lstlisting}
\begin{exe}
  \ex
  \gll Jeder Bauer, der einen Esel besitzt, schlägt ihn. \\
  every farmer that a.\textsc{acc} donkey owns beats it.\textsc{acc}\\
  \glt `Every farmer who owns a donkey beats it.'  \hfill \citep{Geach62}
\end{exe} 
\end{lstlisting}


%\outputbox{
\begin{exe}
	\ex
	\gll Jeder Bauer, der einen Esel besitzt, schlägt ihn. \\
	every farmer that a.\textsc{acc} donkey owns beats it.\textsc{acc}\\
	\glt `Every farmer who owns a donkey beats it.'  \hfill \citep{Geach62}
\end{exe} 
%}

\end{frame}


%%%%%%%%%%%%%%%%%%%%%%%%%%%%%%%%%%
\begin{frame}[fragile]

%\frametitle{Glossieren}

Use \textbf{curly brackets} to group elements that are being \textbf{glossed as a unit}.

\begin{lstlisting}
\begin{exe}
  \ex
    \gll {Multiword expression} -s can be glossed too.\\
    Mehrwortlexem -e.\textsc{pl} können sein glossiert auch\\
    \glt `Auch Mehrwortlexeme können glossiert werden.'
  \ex 
    \gll Peter$_{1}$ $t_{1}$  schläft$_{2}$  $t_{2}$\\
    Peter {} sleeps\\
    \glt `Peter ist sleeping.'
\end{exe}
\end{lstlisting}


%\outputbox{
\begin{exe}
\ex
  \gll {Multiword expression} -s can be glossed too.\\
Mehrwortlexem -e.\textsc{pl} können sein glossiert auch\\
\glt `Auch Mehrwortlexeme können glossiert werden.'
\ex 
\gll Peter$_{1}$ $t_{1}$  schläft$_{2}$  $t_{2}$\\
Peter {} sleeps\\
\glt `Peter ist sleeping.'
\end{exe}
%}

%\begin{itemize}
%\item[\ras] 
\hfill \emph{Leipzig Glossing Rules} \citep[cf.][]{LeipzigGloss15a}
%\end{itemize}

\end{frame}


%%%%%%%%%%%%%%%%%%%%%%%%%%%%%%%%%%%
%%%%%%%%%%%%%%%%%%%%%%%%%%%%%%%%%%%
%\subsubsection{Verweise auf Beispiele}
%%\frame{
%%	\frametitle{~}
%%	\begin{multicols}{2}
%%		\tableofcontents[currentsection,hideallsubsections]
%%	\end{multicols}
%%}
%%%%%%%%%%%%%%%%%%%%%%%%%%%%%%%%%%%
%%%%%%%%%%%%%%%%%%%%%%%%%%%%%%%%%%%
%
%\begin{frame}[fragile]
%\frametitle{Verweise auf Beispiele}
%
%Mit den bereits eingeführten Befehlen: \lstinline|\label{}| und \lstinline|\ref{}| 
%{\footnotesize
%\begin{lstlisting}
%\begin{exe}
%\ex \label{ex:Bsp1}
%\begin{xlist}
%\ex \label{ex:Bsp2}
%\gll Auch Mehrwortelemente könn-en glossiert werden.\\
%also {more.word.elements} can-\textsc{3.pl} glossed be\\
%\ex[*]{Das ungrammatischer ist Beispiel.}\label{ex:Bsp3}
%\ex \label{ex:Bsp4}
%\begin{xlist}
%\ex[]{das grammatische Beispiel}\label{ex:Bsp5}
%\ex[]{noch ein grammatisches Beispiel}\label{ex:Bsp6}
%\end{xlist}	
%\end{xlist}
%\end{exe}
%Die Beispiele (\ref{ex:Bsp1}), (\ref{ex:Bsp2}), (\ref{ex:Bsp3}),
%(\ref{ex:Bsp4}), (\ref{ex:Bsp5}) und (\ref{ex:Bsp6}) zeigen die 
%Verwendung von Verweisen auf Beispiele.
%\end{lstlisting}
%}
%\end{frame}
%
%
%%%%%%%%%%%%%%%%%%%%%%%%%%%%%%%%%%%
%
%\begin{frame}
%
%\begin{exe}
%\ex \label{ex:Bsp1}
%\begin{xlist}
%\ex \label{ex:Bsp2}
%\gll Auch Mehrwortelemente könn-en glossiert werden.\\
%also {more.word.elements} can-\textsc{3.pl} glossed be\\
%\ex[*]{Das ungrammatischer ist Beispiel.}\label{ex:Bsp3}
%\ex \label{ex:Bsp4}
%\begin{xlist}
%\ex[]{das grammatische Beispiel}\label{ex:Bsp5}
%\ex[]{noch ein grammatisches Beispiel}\label{ex:Bsp6}
%\end{xlist}	
%\end{xlist}
%\end{exe}
%Die Beispiele (\ref{ex:Bsp1}), (\ref{ex:Bsp2}), (\ref{ex:Bsp3}), (\ref{ex:Bsp4}), (\ref{ex:Bsp5}) und (\ref{ex:Bsp6}) zeigen die Verwendung von Verweisen auf Beispiele.
%
%\end{frame}


%%%%%%%%%%%%%%%%%%%%%%%%%%%%%%%%%%
%%%%%%%%%%%%%%%%%%%%%%%%%%%%%%%%%%
\subsubsection{Customizing identifiers}
%\frame{
%	\frametitle{~}
%	\begin{multicols}{2}
%		\tableofcontents[currentsection,hideallsubsections]
%	\end{multicols}
%}
%%%%%%%%%%%%%%%%%%%%%%%%%%%%%%%%%%
%%%%%%%%%%%%%%%%%%%%%%%%%%%%%%%%%%

\begin{frame}[fragile]
\frametitle{Customizing identifiers}

With the command \lstinline|\exi{ }| instead of \ltxterm{ex}, you can choose own identifiers. 
The automatic numbering skips the \ltxterm{exi} examples, see (\ref{ex:DetNP}), %(\ref{ex:DetNPAP}), 
and (\ref{ex:NPDet}).%, and (\ref{ex:DetPPNP}).

\begin{lstlisting}
\begin{exe}
  \ex a noun phrase \label{ex:DetNP}
  \exi{($\alpha$)} a noun phrase modified with a PP \label{ex:DetNPAP}
  \ex[*]{noun phrase a}\label{ex:NPDet}
  \exi{($\beta$)}[*]{a with a PP noun phrase modified} \label{ex:DetPPNP}
\end{exe}
\end{lstlisting}


\begin{exe}
\ex a noun phrase \label{ex:DetNP}
\exi{($\alpha$)} a noun phrase modified with a PP \label{ex:DetNPAP}
\ex[*]{noun phrase a}\label{ex:NPDet}
\exi{($\beta$)}[*]{a with a PP noun phrase modified} \label{ex:DetPPNP}
\end{exe}


\end{frame}


%%%%%%%%%%%%%%%%%%%%%%%%%%%%%%%%%%
\begin{frame}[fragile]

With the command \lstinline|\exr{ }| and \lstinline|\exp{ }| instead of \ltxterm{ex}, you can \textbf{repeat the numbering} of earlier examples or repeat it with a \textbf{prime}, respectively. 

\begin{lstlisting}
\begin{exe}
  \ex[]{a new noun phrase}
  \exr{ex:DetNP}[]{a noun phrase}
  \exp{ex:NPDet}[*]{noun a phrase}
  \ex[]{another noun phrase}
\end{exe}
\end{lstlisting}

\begin{exe}
	\ex[]{a new noun phrase}
	\exr{ex:DetNP}[]{a noun phrase}
	\exp{ex:NPDet}[*]{noun a phrase}
	\ex[]{another noun phrase}
\end{exe}


For further features of \ltxpack{gb4e}, take a look at the package documentation \citep{Kolb&Co10a}.

\end{frame}


%%%%%%%%%%%%%%%%%%%%%%%%%%%%%%%%%%
%%%%%%%%%%%%%%%%%%%%%%%%%%%%%%%%%%
\section{langsci-gb4e}
\frame{
	\frametitle{~}
	%	\begin{multicols}{2}
	\tableofcontents[currentsection,hideallsubsections]
	%	\end{multicols}
}
%%%%%%%%%%%%%%%%%%%%%%%%%%%%%%%%%%
%%%%%%%%%%%%%%%%%%%%%%%%%%%%%%%%%%

\begin{frame}[fragile]
\frametitle{langsci-gb4e}

There is an \textbf{improved version} of \ltxpack{gb4e} made by the \emph{Language Science Press} team.

\begin{lstlisting}
\usepackage{langsci-gb4e}
\end{lstlisting}

You can use \textbf{all \ltxpack{gb4e} commands}, but there is also a shorter version for the \ltxterm{exe} and \ltxterm{xlist} environments: open and close every level with \lstinline|\ea| and \lstinline|\z|, every further item in a level is given by \lstinline|\ex|. 

\begin{multicols}{2}
	
%{\footnotesize 
\begin{lstlisting}
\ea This is an example. 
  \ea This is an example.
  \ex This is an example. 
  \z		
\ex This is an example.
\z 
\end{lstlisting}
%}

%\outputbox{
\ea This is an example. 
\ea This is an example.
\ex This is an example. 
\z		
\ex This is an example.
\z 
%} 

\end{multicols}

For further features of \ltxpack{gb4e}, take a look at the LSP guidelines \citep{Nordhoff&Co}.


\end{frame}


%%%%%%%%%%%%%%%%%%%%%%%%%%%%%%%%%%
%%%%%%%%%%%%%%%%%%%%%%%%%%%%%%%%%%
\section{jambox}
\frame{
	\frametitle{~}
	%	\begin{multicols}{2}
	\tableofcontents[currentsection,hideallsubsections]
	%	\end{multicols}
}
%%%%%%%%%%%%%%%%%%%%%%%%%%%%%%%%%%
%%%%%%%%%%%%%%%%%%%%%%%%%%%%%%%%%%

\begin{frame}[fragile]
\frametitle{jambox}

For adding comments to your examples, you can use the \lstinline|\hfill| command (alignment on the right side)

\begin{lstlisting}
\ea  If no mistake have you made, yet losing you are \dots\ a different game you should play. \hfill [Yoda-English]
\ex Das ist ein Beispiel. \hfill [German]
\ex Este es un ejemplo. \hfill [Spanish]
\z 

\end{lstlisting}
%\settowidth\jamwidth{ Test} 
\ea If no mistake have you made, yet losing you are \dots\ a different game you should play. \hfill [Yoda-English]

\ex Das ist ein Beispiel. \hfill [German]

\ex Este es un ejemplo. \hfill [Spanish]
\z 

\end{frame}


%%%%%%%%%%%%%%%%%%%%%%%%%%%%%%%%%%
\begin{frame}[fragile]

\dots\ or the \ltxpack{jambox} package.

\smallskip

You will need the \textbf{\ltxterm{jambox.sty} file} in the same folder as your \ltxterm{.tex} file.

\begin{lstlisting}
\usepackage{jambox}
\end{lstlisting}

\smallskip

\pause 

With \ltxterm{jambox.sty} you can adjust the \textbf{spacing from the right margin to the begin of the comments} (see: \lstinline|\settowidth\jamwidth{ }|)

\begin{lstlisting}
\settowidth\jamwidth{[Yoda-English]X} 
\ea If no mistake have you made, yet losing you are \dots\ a different game you should play. \jambox{[Yoda-English]}
\ex Das ist ein Beispiel. \jambox{[German]}
\ex Este es un ejemplo. \jambox{[Spanish]}
\z 
\end{lstlisting}

\settowidth\jamwidth{[Yoda-English]X} 
\ea If no mistake have you made, yet losing you are \dots\ a different game you should play. \jambox{[Yoda-English]}
\ex Das ist ein Beispiel. \jambox{[German]}
\ex Este es un ejemplo. \jambox{[Spanish]}
\z 

\end{frame}


%%%%%%%%%%%%%%%%%%%%%%%%%%%%%%%%%%
\begin{frame}[fragile]

Complex examples:

\begin{lstlisting}
\settowidth\jamwidth{(Chomsky, 1957: 1)XX} 

\ea[?]{Patience you must have, my young padawan. \jambox{(Yoda, 2005)}
}\label{ex:Jam1}

\ex[]{Syntax is the study of the principles and processes by which sentences are constructed in particular languages. Syntactic investigation of a given language has as its goal the construction of a grammar that can be viewed as a device of some sort for producing the sentences of the language under analysis. \jambox{\citep[1]{Chomsky57a}}
}\label{ex:Jam2}	

\ex[]{
\gll Jeder Bauer, der einen Esel besitzt, schlägt ihn. \\ 
every farmer that a.\textsc{m}.\textsc{acc} donkey owns beats it.\textsc{m}.\textsc{acc} \\
\glt `Every farmer who owns a donkey beats it.' 
\jambox{\citep{Geach62}}
}\label{ex:Jam3}  
\z 

\end{lstlisting}

\end{frame}


%%%%%%%%%%%%%%%%%%%%%%%%%%%%%%%%%%
\begin{frame}[fragile]

\settowidth\jamwidth{(Chomsky, 1957: 1)XX} 
\ea[?]{Patience you must have, my young padawan. \jambox{(Yoda, 2005)}
	}\label{ex:Jam1}

\ex[]{Syntax is the study of the principles and processes by which sentences are constructed in particular languages. Syntactic investigation of a given language has as its goal the construction of a grammar that can be viewed as a device of some sort for producing the sentences of the language under analysis. \jambox{\citep[1]{Chomsky57a}}
	}\label{ex:Jam2}	

\ex[]{
	\gll Jeder Bauer, der einen Esel besitzt, schlägt ihn. \\ 
	every farmer that a.\textsc{m}.\textsc{acc} donkey owns beats it.\textsc{m}.\textsc{acc} \\
	\glt `Every farmer who owns a donkey beats it.' 
	\jambox{\citep{Geach62}}
}\label{ex:Jam3}  
\z 

\end{frame}

%%%%%%%%%%%%%%%%%%%%%%%%%%%%%%%%%
\begin{frame}[fragile]
\frametitle{Exercise}


Go to \url{https://github.com/langsci/latex4linguists/blob/master/3-2.md}\\
and follow the instructions of \textbf{all blocks} in your \texttt{.tex} file.

%Download the PDF \alert{\texttt{myDocument-EX4.pdf}} and replicate it with the commands you have already learnt. Follow the instructions in the last section and install the packages.

\end{frame}


%%%%%%%%%%%%%%%%%%%%%%%%%%%%%%%%%%%
%%%%%%%%%%%%%%%%%%%%%%%%%%%%%%%%%%%
%\section{XY}
%%\frame{
%%\begin{multicols}{2}
%%\frametitle{~}
%%	\tableofcontents[currentsection]
%%\end{multicols}
%%}
%%%%%%%%%%%%%%%%%%%%%%%%%%%%%%%%%%%
%
%\begin{frame}{XY}
%
%\begin{itemize}
%	\item XY
%\end{itemize}
%
%\end{frame}


%%%%%%%%%%%%%%%%%%%%%%%%%%%%%%%%%%%%
%%%%%%%%%%%%%%%%%%%%%%%%%%%%%%%%%%%%
%\iftoggle{handout}{
%%% BEGIN handout true
%
%%%%%%%%%%%%%%%%%%%%%%%%%%%%%%%%%%%%
%	
%%Test Toggle ON
%
%}
%%% END handout true 
%%% BEGIN handout false
%{
%%%%%%%%%%%%%%%%%%%%%%%%%%%%%%%%%%%%
%
%% Test Toggle OFF
%
%}%% END handout false
%%%%%%%%%%%%%%%%%%%%%%%%%%%%%%%%%%%%