%%%%%%%%%%%%%%%%%%%%%%%%%%%%%%%%%%%%%%%%%%%%%%%%
%% Compile the master file!
%% 		Slides: Antonio Machicao y Priemer
%% 		Course: Wissenschaftliches Arbeiten
%%%%%%%%%%%%%%%%%%%%%%%%%%%%%%%%%%%%%%%%%%%%%%%%


%%%%%%%%%%%%%%%%%%%%%%%%%%%%%%%%%%%%%%%%%%%%%%%%%%%%
%%%             Metadata                         
%%%%%%%%%%%%%%%%%%%%%%%%%%%%%%%%%%%%%%%%%%%%%%%%%%%%      

\title{
	Wissenschaftliches Arbeiten in der Linguistik\\
	(Technische Übung)
}

\subtitle{Formelles}

\author[aMyP]{
	{\small Antonio Machicao y Priemer}
	\\
	{\footnotesize \url{www.linguistik.hu-berlin.de/staff/amyp}}
%	\\
%	{\footnotesize \href{mailto:mapriema@hu-berlin.de}{mapriema@hu-berlin.de}}
}

\institute{Institut für deutsche Sprache und Linguistik}

\date{ }

%\publishers{\textbf{6. linguistischer Methodenworkshop \\ Humboldt-Universität zu Berlin}}

%\hyphenation{nobreak}


%%%%%%%%%%%%%%%%%%%%%%%%%%%%%%%%%%%%%%%%%%%%%%%%%%%%
%%%             Preamble's End                   
%%%%%%%%%%%%%%%%%%%%%%%%%%%%%%%%%%%%%%%%%%%%%%%%%%%%      


%%%%%%%%%%%%%%%%%%%%%%%%%%%%%%%%%%%
%%%%%%%%%%%%%%%%%%%%%%%%%%%%%%%%%%%    
%% Title slide 
\begin{frame}
  \HUtitle
\end{frame}


%% Contents slide
\frame{
\begin{multicols}{2}
	\frametitle{Inhaltsverzeichnis}
	\tableofcontents
%		[hideallsubsections]
%		[pausesections]
\end{multicols}
	}


%%%%%%%%%%%%%%%%%%%%%%%%%%%%%%%%%%%%
%%%%%%%%%%%%%%%%%%%%%%%%%%%%%%%%%%%%
%% Extra literature

\nocite{Freitag&MyP15a}
%\nocite{Knuth1986}
%\nocite{Kopka94a}
\nocite{MyP17c}

%%%%%%%%%%%%%%%%%%%%%%%%%%%%%%%%%%%%
%%%%%%%%%%%%%%%%%%%%%%%%%%%%%%%%%%%%


%%%%%%%%%%%%%%%%%%%%%%%%%%%%%%%%%%%%
%%%%%%%%%%%%%%%%%%%%%%%%%%%%%%%%%%%%
%%% Basic literature for these slides
%
%\begin{frame}
%\frametitle{Grundlage \& empfohlene Lektüre}
%
%\dots basierend auf \citet{Freitag&MyP15a} und auf \citet{MyP&Kerkhof16a}\\
%\ras \href{https://www.researchgate.net/publication/279514740_LATEX-Einfuhrung_fur_Linguisten}{LINK}
%
%\end{frame}


%%%%%%%%%%%%%%%%%%%%%%%%%%%%%%%%%%%%
%%%%%%%%%%%%%%%%%%%%%%%%%%%%%%%%%%%%
\section{Kontakt}
%\frame{
%\begin{multicols}{2}
%\frametitle{~}
%	\tableofcontents[currentsection]
%\end{multicols}
%}
%
%%%%%%%%%%%%%%%%%%%%%%%%%%%%%%%%%%%%

\begin{frame}{Kontakt}

\begin{itemize}
	\item \textbf{Dozent:} Antonio Machicao y Priemer \textipa{[ma.\t{tS}i."ka.o."Pi."pKi:.m5]}
	\item \textbf{Büro:} Dorotheenstraße 24, Raum: 3.305
	\item \textbf{Telefon:} +49 (30) 2093-9702
	\item \textbf{Webseite:} \href{www.linguistik.hu-berlin.de/staff/amyp}{www.linguistik.hu-berlin.de/staff/amyp}
	\item \textbf{E-Mail:} \href{mailto:mapriema@hu-berlin.de}{mapriema@hu-berlin.de}
	\item[]
	\item \textbf{Sprechstunde}: Mo. 10--12h (Anmeldung per E-Mail erforderlich!)
	
	\textbf{Keine Sprechstunden}: 14.01.--20.01.
\end{itemize}	


\end{frame}


%%%%%%%%%%%%%%%%%%%%%%%%%%%%%%%%%%%%
%%%%%%%%%%%%%%%%%%%%%%%%%%%%%%%%%%%%
\section{Sekretariat}
%\frame{
%\begin{multicols}{2}
%\frametitle{~}
%	\tableofcontents[currentsection]
%\end{multicols}
%}
%%%%%%%%%%%%%%%%%%%%%%%%%%%%%%%%%%%%

\begin{frame}{Sekretariat}

\begin{itemize}
	\item[] \textbf{Anina Klein}	
	\item \textbf{Büro:} Dorotheenstraße 24, Raum: 3.306
	\item \textbf{Telefon:} +49 (30) 2093-9639
	\item \textbf{E-Mail:} \href{mailto:Anina.Klein@cms.hu-berlin.de}{Anina.Klein@cms.hu-berlin.de}
\end{itemize}	

\end{frame}


%%%%%%%%%%%%%%%%%%%%%%%%%%%%%%%%%%%%
%%%%%%%%%%%%%%%%%%%%%%%%%%%%%%%%%%%%
\section{Moodle}	
%\frame{
%\begin{multicols}{2}
%\frametitle{~}
%	\tableofcontents[currentsection]
%\end{multicols}
%}
%%%%%%%%%%%%%%%%%%%%%%%%%%%%%%%%%%%%

\begin{frame}{Moodle}

\begin{itemize}
	\item Alle \textbf{Folien} und \textbf{Materialien} werden über Moodle zur Verfügung gestellt.
	\item[]
	\item Wichtige \textbf{Hinweise} (Ausfälle, etc.) werden immer über Moodle bekannt gegeben.
	\item[]
	\item Die \textbf{Hausaufgaben} werden nur über Moodle abgegeben! 
	
	(Achten Sie auf das \textbf{Abgabedatum}!)
	\item[]
	\item \textbf{Moodleseite des Kurses:} \href{https://moodle.hu-berlin.de/course/view.php?id=83977}{https://moodle.hu-berlin.de/course/view.php?id=83977}\\
	\textbf{Moodleschlüssel:} kompilieren
\end{itemize}		

\end{frame}


%%%%%%%%%%%%%%%%%%%%%%%%%%%%%%%%%%%%
%%%%%%%%%%%%%%%%%%%%%%%%%%%%%%%%%%%%
\section{Technische Beratung}
%\frame{
%\begin{multicols}{2}
%\frametitle{~}
%	\tableofcontents[currentsection]
%\end{multicols}
%}
%%%%%%%%%%%%%%%%%%%%%%%%%%%%%%%%%%%%

\begin{frame}{Technische Beratung}

\begin{itemize}
	\item \textbf{Stud. Mitarbeiterin:} Pia Linscheid
	
	\item \textbf{Büro:} Dorotheenstraße 24, Raum: 3.307

	\item \textbf{E-Mail:} \href{mailto:linschep@cms.hu-berlin.de}{linschep@cms.hu-berlin.de}
	
	\item \textbf{Sprechstunde:} Mo., 12--13 (oder nach Vereinbarung)
\end{itemize}	

\end{frame}





%%%%%%%%%%%%%%%%%%%%%%%%%%%%%%%%%%%%
%%%%%%%%%%%%%%%%%%%%%%%%%%%%%%%%%%%%
%\section{Tutorien}
%%\frame{
%%\begin{multicols}{2}
%%\frametitle{~}
%%	\tableofcontents[currentsection]
%%\end{multicols}
%%}
%%%%%%%%%%%%%%%%%%%%%%%%%%%%%%%%%%%%
%
%\begin{frame}{Tutorien}
%
%	\begin{itemize}
%		\item \textbf{Tutorium zum GK Semantik}
%	\end{itemize}	
%	
%	\scalebox{0.8}{
%		\begin{tabular}{p{5cm}l}
%			\textbf{Lehrender} & Nico Lehmann \\
%			\textbf{Seminar -- Zeit \& Raum} & Do., 16--18~|~UL 6,1070 \\
%			\textbf{Moodle}	&  \href{https://moodle.hu-berlin.de/course/view.php?id=76458}{https://moodle.hu-berlin.de/course/view.php?id=76458}\\
%			& Wahrheit \\
%		\end{tabular}
%	}
%
%	\begin{itemize}
%		\item \textbf{Tutorium zur VL Syntax}
%	\end{itemize}
%	
%	\scalebox{0.8}{	
%		\begin{tabular}{p{5cm}l}
%			\textbf{Lehrende} & Luise Hiller \& Robert Fritzsche \\
%			\textbf{Seminar -- Zeit \& Raum} & Mi., 18--20~|~DOR 24, 1.102 \\
%			\textbf{Moodle}	&  \href{https://moodle.hu-berlin.de/course/view.php?id=76438}{https://moodle.hu-berlin.de/course/view.php?id=76438}\\
%			& Tochterkopf \\
%		\end{tabular}
%	}
%
%\vspace{2mm}
%Die Tutorien beginnen \textbf{erst in der zweiten Vorlesungswoche}!
%	
%\end{frame}


%%%%%%%%%%%%%%%%%%%%%%%%%%%%%%%%%%%%
%%%%%%%%%%%%%%%%%%%%%%%%%%%%%%%%%%%%
\section{Leistungserbringung}
%\frame{
%\begin{multicols}{2}
%\frametitle{~}
%	\tableofcontents[currentsection]
%\end{multicols}
%}
%%%%%%%%%%%%%%%%%%%%%%%%%%%%%%%%%%%%

\begin{frame}{Leistungserbringung}

	\begin{itemize}
		\item Regelmäßige und \textbf{aktive!} Teilnahme
		\item[+]
		
		\item Lektüre
		\item[+]
				
		\item 6 (von 8) Hausaufgaben 
		
		\item[]
		\item[=] \textbf{Voraussetzungen für die Unterschrift}
			
	\end{itemize}
	
	\begin{itemize}
		\item Laut Prüfungsordnung:
		
		\textbf{60} Stunden $=$ 25 Stunden Präsenzzeit $+$ \textbf{35} Stunden Vor- und Nachbereitung
	\end{itemize}	
\end{frame}


%%%%%%%%%%%%%%%%%%%%%%%%%%%%%%%%%%%%
\begin{frame}{Hausaufgaben}

\begin{itemize}

	\item Es gibt insgesamt \textbf{8 mögliche Hausaufgaben}. Sie müssen \textbf{6 Hausaufgaben} erledigen.

	\begin{itemize}
		\item \textbf{HA0:} (u.\,a.) Bereiten Sie Ihr System für \LaTeX\ vor (bzw.\ richten Sie sich ein Overleaf-Konto ein) und testen Sie die \texttt{Testdatei.tex}.
		
		\item \textbf{HA1--HA7:} Die Fragen sind in den Foliensätzen. Die Lösungen müssen rechtzeitig über Moodle hochgeladen werden.
	\end{itemize}

	\item \textbf{Abgabedatum} der Hausaufgaben \ras Semesterplan
	
	\item Die Hausaufgaben werden \textbf{nur über Moodle angenommen}.
	
	\item Sie können Ihre Hausaufgaben \textbf{in Gruppen} lösen, die Abgabe erfolgt aber \textbf{individuell}!

\end{itemize}

\end{frame}


%%%%%%%%%%%%%%%%%%%%%%%%%%%%%%%%%%%%
\begin{frame}

Was wird noch erwartet\dots

\begin{itemize}
	\item Befassen Sie sich mit der angegebenen \textbf{Literatur}.
	\item[]
	\item Fragen und Diskussionen
	\item[]	
	\item Während des Seminars bitte kein Facebook, Twitter, Tumblr,
	Youtube, WhatsApp, Netflix, \dots
	\item[]	
	\item Wenn Sie früher gehen müssen, sagen Sie mir bitte zu Beginn der
	Stunde Bescheid, und \textbf{nehmen Sie bitte Platz in der Nähe der Tür}.
	
\end{itemize}

\end{frame}


%%%%%%%%%%%%%%%%%%%%%%%%%%%%%%%%%%%%
%%%%%%%%%%%%%%%%%%%%%%%%%%%%%%%%%%%%
\section{Kurse im Career Center der HU}
%\frame{
%\begin{multicols}{2}
%\frametitle{~}
%	\tableofcontents[currentsection]
%\end{multicols}
%}
%%%%%%%%%%%%%%%%%%%%%%%%%%%%%%%%%%%%

\begin{frame}{Kurse im Career Center der HU}

\begin{itemize}
	\item Wissenschaftliches oder journalistisches Schreiben, Rhetorik und Kommunikation, wirkungsvolles Präsentieren \textbf{und vieles mehr} können Sie dort lernen.
	
	\item[]
	
	\item[] \url{https://www.hu-berlin.de/de/career-center}
	
\end{itemize}		

\end{frame}


%%%%%%%%%%%%%%%%%%%%%%%%%%%%%%%%%%%%%
%%%%%%%%%%%%%%%%%%%%%%%%%%%%%%%%%%%%%
%\section{Lesekreis}
%%\frame{
%%\begin{multicols}{2}
%%\frametitle{~}
%%	\tableofcontents[currentsection]
%%\end{multicols}
%%}
%%%%%%%%%%%%%%%%%%%%%%%%%%%%%%%%%%%%%
%
%\begin{frame}{Lesekreis}
%
%\begin{itemize}
%	\item Wollen Sie mit anderen Studenten linguistische Texte lesen und diskutieren?
%	
%	\item[]
%	
%	\item Sehr empfehlenswert, wenn Sie einen schwierigen Text lesen müssen, ihn aber nicht alleine bearbeiten wollen/können!
%	
%	\item[]
%	
%	\item \textbf{Organisation:}  Tjona Kristina Sommer
%	
%	\item \textbf{E-Mail:} \href{mailto:sommerkr@hu-berlin.de}{sommerkr@hu-berlin.de}
%	
%	\item \textbf{Webseite:} 
%	 \url{www.klesekreis.junge-sprachwissenschaft.de}
%	
%\end{itemize}		
%
%\end{frame}


%%%%%%%%%%%%%%%%%%%%%%%%%%%%%%%%%%%%
%%%%%%%%%%%%%%%%%%%%%%%%%%%%%%%%%%%%
\section{Kolloquien}
%\frame{
%\begin{multicols}{2}
%\frametitle{~}
%	\tableofcontents[currentsection]
%\end{multicols}
%}
%%%%%%%%%%%%%%%%%%%%%%%%%%%%%%%%%%%%

\begin{frame}{Kolloquien}

\begin{itemize}
	\item Es gibt zwei Kolloquien am IdSL:

	\begin{itemize}
		\item Kolloquium für \textbf{Korpuslinguistik und Phonetik/Phonologie}
		\item[]

		\item Kolloquium für \textbf{Syntax \& Semantik}
	\end{itemize}

	\item \textbf{Kolloquium Syntax \& Semantik} (2 SWS; 0 LP)
	
	\item[] Leitung: Elisabeth Verhoeven \& Stefan Müller
	
	\item[] Ort \& Zeit: Mo., 16:00--17:30 $|$ DOR 24, 1.401 
	
	\item[]
	\item Sie können \textbf{Ihre Arbeit präsentieren/diskutieren}!

	\item[]
	\item Sie können das Kolloquium auch \textbf{einfach nur aus Interesse} besuchen!
\end{itemize}		

\end{frame}


%%%%%%%%%%%%%%%%%%%%%%%%%%%%%%%%%%%%
%%%%%%%%%%%%%%%%%%%%%%%%%%%%%%%%%%%%
\section{Versuchspersonendatenbank LingEx}
%\frame{
%\begin{multicols}{2}
%\frametitle{~}
%	\tableofcontents[currentsection]
%\end{multicols}
%}
%%%%%%%%%%%%%%%%%%%%%%%%%%%%%%%%%%%%

\begin{frame}{Versuchspersonendatenbank LingEx}

\begin{itemize}
	\item Wollen Sie \textbf{an linguistischen Experimenten teilnehmen}?
	\item Helfen Sie Ihren Mitstudenten/Kollegen! Helfen Sie der Forschung!
	
	\item[]
	
	\item Da die \textbf{Datenbank relativ neu} ist, verfügt Sie leider noch nicht über allzu viele registrierte Probanden, weshalb es für die Durchführung von Experimenten sehr wichtig und hilfreich wäre, wenn sich noch \textbf{möglichst viele} Studierende anmelden.
	
	\item[] 

	\item \textbf{Webseite:}  \url{https://lingex.zas.gwz-berlin.de/public/}	
	
	\item[]
	\item Falls Sie selbst Interesse haben, Experimente über LingEx zu bewerben, wenden Sie sich bitte an die ZAS-Mitarbeiter.
	
	\item \textbf{E-Mail:} \href{mailto:salfner@leibniz-zas.de}{salfner@leibniz-zas.de}
		
\end{itemize}		

\end{frame}


%%%%%%%%%%%%%%%%%%%%%%%%%%%%%%%%%%%%
%%%%%%%%%%%%%%%%%%%%%%%%%%%%%%%%%%%%
\section{Language Science Press}
%\frame{
%\begin{multicols}{2}
%\frametitle{~}
%	\tableofcontents[currentsection]
%\end{multicols}
%}
%%%%%%%%%%%%%%%%%%%%%%%%%%%%%%%%%%%%

\begin{frame}{Language Science Press}

\begin{itemize}
	
	\item Open Access-Verlag an der HU-Berlin
	\item[]

	\item \textbf{Webseite:}  \url{http://langsci-press.org/}
	\item[]

	\item \textbf{Aktuelle (gratis) Bücher:} \url{http://langsci-press.org/catalog}
	\item[]	

	\item \textbf{Unterstützer:}
	\url{http://langsci-press.org/supporters}
	\item[]	
	
	\item Werden Sie Unterstützer und Proof-reader!
		
\end{itemize}		

\end{frame}


%%%%%%%%%%%%%%%%%%%%%%%%%%%%%%%%%%%%
%%%%%%%%%%%%%%%%%%%%%%%%%%%%%%%%%%%%
\section{Literatur für das Seminar}
%\frame{
%\begin{multicols}{2}
%\frametitle{~}
%	\tableofcontents[currentsection]
%\end{multicols}
%}
%%%%%%%%%%%%%%%%%%%%%%%%%%%%%%%%%%%%

\begin{frame}{Literatur für das Seminar}

\begin{itemize}
	\item Für jede Sitzung wird die Literatur aus dem Semesterplan (s.~Handout bzw.\ Semesterplan in Moodle) vorausgesetzt.
	
	\item Die Lektüre für jede Sitzung wird als PDF über Moodle bereitgestellt.
	\item[]
	\item Dieser Kurs basiert hauptsächlich auf \citet{Krifka08a, Krifka13a, Luedeling12a, Freitag&MyP15a, MyP17c, Meindl11a, Rothstein11a, Albert&Marx10a}.

\end{itemize}		

\end{frame}


%%%%%%%%%%%%%%%%%%%%%%%%%%%%%%%%%%%%
\begin{frame}

\begin{itemize}
	\item Die \textbf{Lexika} \citet{Glueck&Roedel16a} und \citet{Schierholz&Co18} sind durch die Webseite der HU-Bibliothek kostenlos konsultierbar.
	
	\item Die \textbf{\LaTeX -Materialien} \citep{Freitag&MyP15a} und die \textbf{Hinweise für Seminararbeiten} \citep{MyP17c} können Sie aus den angegebenen URLs herunterladen.
	
	\item Die \textbf{Bücher} von \citet{Albert&Marx10a}, \citet{Meindl11a} und \citet{Rothstein11a} sind über die Webseite der HU-Bibliothek kostenlos herunterladbar.
\end{itemize}
\end{frame}


%%%%%%%%%%%%%%%%%%%%%%%%%%%%%%%%%%%
%%%%%%%%%%%%%%%%%%%%%%%%%%%%%%%%%%%
\section{Verbesserungsvorschläge}
%\frame{
%\begin{multicols}{2}
%\frametitle{~}
%	\tableofcontents[currentsection]
%\end{multicols}
%}
%%%%%%%%%%%%%%%%%%%%%%%%%%%%%%%%%%%

\begin{frame}
\frametitle{Beschwerden, Verbesserungsvorschläge}

\begin{itemize}
	\item mündlich
	\item per Mail oder 
	\item anonym durch die Lehrevaluation am Ende des Semesters
\end{itemize}
\end{frame}


%%%%%%%%%%%%%%%%%%%%%%%%%%%%%%%%%%%
%%%%%%%%%%%%%%%%%%%%%%%%%%%%%%%%%%%
\section{Mailverkehr}
%\frame{
%\begin{multicols}{2}
%\frametitle{~}
%	\tableofcontents[currentsection]
%\end{multicols}
%}
%%%%%%%%%%%%%%%%%%%%%%%%%%%%%%%%%%%

\begin{frame}
\frametitle{Mailverkehr}

\begin{itemize}
	\item HU-Mail-Adresse verwenden (wegen Spam-Gefahr)
	
	\item Geben Sie Ihren Vor- und Nachname richtig an.
	
	\item Überlegen Sie zuerst, ob Ihre Kommilitonen Ihnen nicht eine Antwort auf die Frage geben können
	
	\ras Benutzen Sie das \textbf{Forum} auf unserer Moodleseite, auch für Fragen bzgl. der Hausaufgaben.
	
	\item Bitte beachten Sie die gängigen Höflichkeitsregeln beim Mailverkehr!
\end{itemize}

\end{frame}


%%%%%%%%%%%%%%%%%%%%%%%%%%%%%%%%%%%%
%%%%%%%%%%%%%%%%%%%%%%%%%%%%%%%%%%%%
\section{Ziel des Kurses}
%\frame{
%\begin{multicols}{2}
%\frametitle{~}
%	\tableofcontents[currentsection]
%\end{multicols}
%}
%%%%%%%%%%%%%%%%%%%%%%%%%%%%%%%%%%%%

\begin{frame}{Ziel des Kurses}
Es handelt sich \textbf{nicht} um eine \textbf{fachliche} Einführung in die Linguistik! ($\rightarrow$~Grundkurs Linguistik \& Übung deutsche Grammatik)

\begin{itemize}
	\item<2-> (Einige wenige) Grundbegriffe der Wissenschaftstheorie\\
	($\rightarrow$ \textbf{Theorie})

	\item<3-> Wie bereite ich Präsentationsformen vor (Seminararbeiten, Referate, etc.)?\\
	($\rightarrow$ \textbf{Form})
	
	\item<4-> Wie gehe ich mit der Recherche und mit der Literatur um?\\
	($\rightarrow$ \textbf{Informationssuche} \&  \textbf{-rezeption})
	
	\item<5-> Wie bereite ich meine Fragestellung vor und wie sammle ich meine Daten?\\
	($\rightarrow$ \textbf{Empirie})
\end{itemize}
\end{frame}


%%%%%%%%%%%%%%%%%%%%%%%%%%%%%%%%%%%%
\begin{frame}

Darüber hinaus:

\begin{itemize}
	
	\item<2-> Wie präsentiere ich meine Arbeit möglichst makellos?\\
	(\ras linguistische und typographische \textbf{Konventionen})
	
	\item<3-> Präsentationssoftware\\
	(\ras \LaTeX )
	 
	\item<4-> Bibliographiesoftware\\
	(\ras EndNote)
	
	\item<5-> Wie baue ich meine Argumentation möglichst logisch auf?\\
	(\ras Logik und Rhetorik)

\end{itemize}			
\end{frame}


%%%%%%%%%%%%%%%%%%%%%%%%%%%%%%%%%%%%
%%%%%%%%%%%%%%%%%%%%%%%%%%%%%%%%%%%%
\section{Hausaufgabe 0}
\frame{
	\begin{multicols}{2}
		%\frametitle{~}
		\tableofcontents[currentsection,hideallsubsections]
	\end{multicols}
}
%%%%%%%%%%%%%%%%%%%%%%%%%%%%%%%%%%%%

\begin{frame}{Hausaufgabe 0: \LaTeX -Vorbereitung}

\begin{itemize}
	\item Laden Sie \ltxpack{MiKTeX} und \ltxpack{TeXstudio} \gs{wie in der Anleitung in Moodle angegeben} herunter.
	\item[]
	
	\item Installieren Sie beide Programme.
	\item[]
	
	\item Folgen Sie dabei der Anleitung in Moodle.
	\item[]
	
	\item Falls Sie \textbf{Probleme bei der Installation} haben, melden Sie sich bitte bei Pia Linscheid \emph{vor} der nächsten Sitzung! Andernfalls werden Sie die kommenden Hausaufgaben nicht abgeben können.
	
	\item Anstatt die Programme zu installieren, können Sie versuchen Ihre Hausaufgaben mit Overleaf zu lösen (Siehe Anleitung in Moodle).
\end{itemize}

\end{frame}


%%%%%%%%%%%%%%%%%%%%%%%%%%%%%%%%%%%%
\begin{frame}{Hausaufgabe 0: Lektüre}

\begin{itemize}
	\item Lesen Sie den LingStudi-Guide (s.~Moodle/Allgemeines)
\end{itemize}

\end{frame}


%%%%%%%%%%%%%%%%%%%%%%%%%%%%%%%%%%%%
\begin{frame}{Hausaufgabe 0: Mitgestalten}

\begin{itemize}
	\item Schreiben Sie bei Moodle im Bereich \gqq{Was will ich in diesem Kurs lernen?} einen bis zwei Stichpunkte auf. Erklären Sie kurz \gs{wenn nötig} was Sie damit meinen.
	
\end{itemize}

\end{frame}
