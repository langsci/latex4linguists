%%%%%%%%%%%%%%%%%%%%%%%%%%%%%%%%%%%%%%%%%%%%%%%%
%% Compile the master file!
%% 		Slides: Antonio Machicao y Priemer
%% 		Course: Wissenschaftliches Arbeiten
%%%%%%%%%%%%%%%%%%%%%%%%%%%%%%%%%%%%%%%%%%%%%%%%


%%%%%%%%%%%%%%%%%%%%%%%%%%%%%%%%%%%%%%%%%%%%%%%%%%%%
%%%             Metadata                         
%%%%%%%%%%%%%%%%%%%%%%%%%%%%%%%%%%%%%%%%%%%%%%%%%%%%  

\title{
	\LaTeX\ for Linguists
}

\subtitle{\LaTeX\ 5: IPA \& verbatim}

\author[aMyP]{
	{\small Sebastian Nordhoff \& Antonio Machicao y Priemer}
	\\
	{\footnotesize \url{www.linguistik.hu-berlin.de/staff/amyp}}
	%	\\
	%	{\footnotesize \href{mailto:mapriema@hu-berlin.de}{mapriema@hu-berlin.de}}
}

\institute{LOT 2019, Amsterdam}

%\date{ }

%\publishers{\textbf{6. linguistischer Methodenworkshop \\ Humboldt-Universität zu Berlin}}

%\hyphenation{nobreak}


%%%%%%%%%%%%%%%%%%%%%%%%%%%%%%%%%%%%%%%%%%%%%%%%%%%%
%%%             Preamble's End                   
%%%%%%%%%%%%%%%%%%%%%%%%%%%%%%%%%%%%%%%%%%%%%%%%%%%%      


%%%%%%%%%%%%%%%%%%%%%%%%%%%%%%%%%%%
%%%%%%%%%%%%%%%%%%%%%%%%%%%%%%%%%%%    
%% Title slide 
\begin{frame}
  \HUtitle
\end{frame}


%% Contents slide
\frame{
%\begin{multicols}{2}
	\frametitle{Contents}
%	\tableofcontents[hideallsubsections]
	\tableofcontents
	%[pausesections]
%\end{multicols}
	}


%%%%%%%%%%%%%%%%%%%%%%%%%%%%%%%%%%%%
%%%%%%%%%%%%%%%%%%%%%%%%%%%%%%%%%%%%
%% Extra literature

\nocite{Freitag&MyP15a}
\nocite{Knuth1986}
\nocite{Kopka94a}
\nocite{MyP17c}
\nocite{MyP&Kerkhof16a}
	
%%%%%%%%%%%%%%%%%%%%%%%%%%%%%%%%%%%%
%%%%%%%%%%%%%%%%%%%%%%%%%%%%%%%%%%%%


%%%%%%%%%%%%%%%%%%%%%%%%%%%%%%%%%%%%%
%%%%%%%%%%%%%%%%%%%%%%%%%%%%%%%%%%%%%
%%%% Basic literature for these slides
%
%\begin{frame}
%\frametitle{Grundlage \& empfohlene Lektüre}
%
%\dots basierend auf \citet{Freitag&MyP15a} und auf \citet{MyP&Kerkhof16a}\\
%\ras \href{https://www.researchgate.net/publication/279514740_LATEX-Einfuhrung_fur_Linguisten}{LINK}
%
%
%\nocite{Kopka94a}
%
%\end{frame}


%%%%%%%%%%%%%%%%%%%%%%%%%%%%%%%%%%
%%%%%%%%%%%%%%%%%%%%%%%%%%%%%%%%%%
\section{Transcriptions with IPA}
\frame{
	\frametitle{~}
	%	\begin{multicols}{2}
	\tableofcontents[currentsection,hideallsubsections]
	%	\end{multicols}
}
%%%%%%%%%%%%%%%%%%%%%%%%%%%%%%%%%%
%%%%%%%%%%%%%%%%%%%%%%%%%%%%%%%%%%

\begin{frame}[fragile]
\frametitle{Transcriptions with IPA}

With Xe\LaTeX , you can use \textbf{Unicode characters} for your transcriptions:

\smallskip

You can copy the Unicode characters for transcriptions from here:\\
\url{http://ipa.typeit.org/full/}
%[ʔ ɛ t͡s ə n d ə ʁ ɐ]

\smallskip

Some fonts cannot display all Unicode characters, \fe try to copy the Unicode characters for the following word and compile using first \ltxpack{lmodern} and then \ltxpack{libertine}.

\ea \textipa{["PE\t{ts}@nd@\textscr{}5]}
\z 

\pause 

\bigskip

The package \textbf{\ltxpack{tipa}} offers commands for transcriptions with IPA, but it is not compatible with all other packages. %\fe with \ltxpack{libertine}

\begin{lstlisting}
\usepackage{tipa}
\end{lstlisting}

%\begin{itemize}
%	\item \ltxpack{tipa} definiert bestimmte \LaTeX -Befehle um. Abhängig von der Font-Kodierung sind manchmal zusätzliche Einstellungen nötig, bspw.\ die Optionen \ltxterm{T3} und \ltxterm{T1} (in dieser Reihenfolge) beim Paket \ltxpack{fontenc} und die Optionen \ltxterm{noenc} und \ltxterm{safe} beim Paket \ltxpack{tipa}.
%\end{itemize}

%\begin{lstlisting}
%\usepackage[T3,T1]{fontenc}
%
%\usepackage[noenc,safe]{tipa}	
%\end{lstlisting}

\end{frame}


%%%%%%%%%%%%%%%%%%%%%%%%%%%%%%%%%%
\begin{frame}[fragile]
%\frametitle{}

\ltxterm{tipa} provides 3 ways to use IPA characters:


\textbf{macros:}

\begin{lstlisting}
[\textglotstop{}an.\textesh{}\textinvscr{}\texttoptiebar{a\textsci{}}.
\textschwa{}n]

[\textsecstress\textepsilon kspl\textschwa \textprimstress ne\textsci\textesh
\textschwa n]
\end{lstlisting}

\vspace{-.25cm}

\begin{multicols}{2}
\ea {[\textglotstop{}an.\textesh{}\textinvscr{}\texttoptiebar{a\textsci{}}.\textschwa{}n]}

\ex {[\textsecstress\textepsilon kspl\textschwa \textprimstress ne\textsci\textesh\textschwa n]}
\z 
\end{multicols}


\textbf{groups of macros:}

\vspace{-.25cm}

\begin{multicols}{2}
\begin{lstlisting}
\textipa{[Pan.SK\t{aI}.@n]} 
\textipa{[""Ekspl@"neIS@n]}
\end{lstlisting}


\ea \textipa{[Pan.SK\t{aI}.@n]}

\ex \textipa{[""Ekspl@"neIS@n]}	
\z 	
\end{multicols}


\textbf{\ltxterm{tipa} environment:}

\vspace{-.25cm}

\begin{multicols}{2}
	
\begin{lstlisting}
\begin{IPA}
[Pan.SK\t{aI}.@n]

[""Ekspl@"neIS@n]
\end{IPA}
\end{lstlisting}


\ea	\begin{IPA}
[Pan.SK\t{aI}.@n]
\end{IPA}

\ex \begin{IPA} 
[""Ekspl@"neIS@n]
\end{IPA}
\z
\end{multicols}

\nocite{Rei04a}

\nocite{Linke05a}

\end{frame}


%%%%%%%%%%%%%%%%%%%%%%%%%%%%%%%%%%%
%\begin{frame}[fragile]
%%\frametitle{IPA-Notation}
%
%\begin{itemize}
%	\item Die IPA-Transkriptionen können in verschiedenen Schriftarten eingebettet werden:
%\end{itemize}
%
%{\scriptsize
%\begin{tabular}{lll}
%	aktuelle        & \lstinline|\textipa{Pan.SK\texttoptiebar{aI}.@n}|          & \textipa{Pan.SK\texttoptiebar{aI}.@n}          \\
%	Standardschrift &                                                            &                                                \\
%	Slanted         & \lstinline|\textsl{\textipa{Pan.SK\texttoptiebar{aI}.@n}}| & \textsl{\textipa{Pan.SK\texttoptiebar{aI}.@n}} \\
%	Bold            & \lstinline|\textbf{\textipa{Pan.SK\texttoptiebar{aI}.@n}}| & \textbf{\textipa{Pan.SK\texttoptiebar{aI}.@n}} \\
%	Sans Serif      & \lstinline|\textsf{\textipa{Pan.SK\texttoptiebar{aI}.@n}}| & \textsf{\textipa{Pan.SK\texttoptiebar{aI}.@n}} \\
%	Typewriter      & \lstinline|\texttt{\textipa{Pan.SK\texttoptiebar{aI}.@n}}| & \texttt{\textipa{Pan.SK\texttoptiebar{aI}.@n}} \\
%\end{tabular}
%}
%
%
%\end{frame}


%%%%%%%%%%%%%%%%%%%%%%%%%%%%%%%%%%%
%\begin{frame}[fragile]
%%\frametitle{IPA-Notation}
%
%\begin{itemize}
%\item Für weitere Features des \ltxpack{tipa}-Pakets schauen Sie sich die Dokumentation an: \citet{Rei04a}
%
%\item Eine gute Auflistung der benötigten Befehle für IPA-Transkriptionen mittels \ltxpack{tipa} finden Sie unter: \citet{Linke05a}
%\end{itemize}
%
%\end{frame}


%%%%%%%%%%%%%%%%%%%%%%%%%%%%%%%%%%
%%%%%%%%%%%%%%%%%%%%%%%%%%%%%%%%%%
\section{Verbatim }
\frame{
	\frametitle{~}
%	\begin{multicols}{2}
		\tableofcontents[currentsection, hideallsubsections]
%	\end{multicols}
}
%%%%%%%%%%%%%%%%%%%%%%%%%%%%%%%%%%

\begin{frame}[fragile]
\frametitle{Verbatim}

If you want to \textbf{write code}, \LaTeX\ provides the \textbf{\ltxterm{verb} command} and the \textbf{\ltxterm{verbatim} environment}.

\begin{multicols}{2}
	
\begin{lstlisting}
\verb|\textbf{test}|

\begin{verbatim}
\textbf{test}
\end{verbatim}

\end{lstlisting}	

\columnbreak

\verb|\textbf{test}|

\begin{verbatim}
\textbf{test}
\end{verbatim}

\end{multicols}

\end{frame}


%%%%%%%%%%%%%%%%%%%%%%%%%%%%%%%%%%
\begin{frame}[fragile]
%\frametitle{Bibliography database}

With the package \ltxpack{listings}, \textbf{more options} for verbatim can be specified:

\begin{lstlisting}
\usepackage{listings}

\lstset{
language=TeX,
backgroundcolor=\color{lightgray},
basicstyle={\footnotesize\ttfamily\color{blue}},
showstringspaces=false,
columns=flexible
}
\end{lstlisting}

\bigskip

This package offers an \textbf{in-line version} with the \lstinline|\lstinline| \textbf{command} 
and the \lstinline|lstlisting| \textbf{environment}.

\bigskip

For the in-line version, use \textbf{characters as delimiters} for your command \lstinline|\lstinline| that are not used in your code.

\end{frame}


%%%%%%%%%%%%%%%%%%%%%%%%%%%%%%%%%
\begin{frame}[fragile]
\frametitle{Exercise}


Go to \url{https://github.com/langsci/latex4linguists/blob/master/3-1.md}\\
and follow the instructions of \textbf{all blocks} in your \texttt{.tex} file.

%Download the PDF \alert{\texttt{myDocument-EX4.pdf}} and replicate it with the commands you have already learnt. Follow the instructions in the last section and install the packages.

\end{frame}


%%%%%%%%%%%%%%%%%%%%%%%%%%%%%%%%%%%
%%%%%%%%%%%%%%%%%%%%%%%%%%%%%%%%%%%
%\section{XY}
%%\frame{
%%\begin{multicols}{2}
%%\frametitle{~}
%%	\tableofcontents[currentsection]
%%\end{multicols}
%%}
%%%%%%%%%%%%%%%%%%%%%%%%%%%%%%%%%%%
%
%\begin{frame}{XY}
%
%\begin{itemize}
%	\item XY
%\end{itemize}
%
%\end{frame}


%%%%%%%%%%%%%%%%%%%%%%%%%%%%%%%%%%%%
%%%%%%%%%%%%%%%%%%%%%%%%%%%%%%%%%%%%
%\iftoggle{handout}{
%%% BEGIN handout true
%
%%%%%%%%%%%%%%%%%%%%%%%%%%%%%%%%%%%%
%	
%%Test Toggle ON
%
%}
%%% END handout true 
%%% BEGIN handout false
%{
%%%%%%%%%%%%%%%%%%%%%%%%%%%%%%%%%%%%
%
%% Test Toggle OFF
%
%}%% END handout false
%%%%%%%%%%%%%%%%%%%%%%%%%%%%%%%%%%%%