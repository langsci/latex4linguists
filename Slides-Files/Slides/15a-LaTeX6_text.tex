%%%%%%%%%%%%%%%%%%%%%%%%%%%%%%%%%%%%%%%%%%%%%%%%
%% Compile the master file!
%% 		Slides: Antonio Machicao y Priemer
%% 		Course: Wissenschaftliches Arbeiten
%%%%%%%%%%%%%%%%%%%%%%%%%%%%%%%%%%%%%%%%%%%%%%%%


%%%%%%%%%%%%%%%%%%%%%%%%%%%%%%%%%%%%%%%%%%%%%%%%%%%%
%%%             Metadata                         
%%%%%%%%%%%%%%%%%%%%%%%%%%%%%%%%%%%%%%%%%%%%%%%%%%%%  

\title{
	\textbf{Wissenschaftliches Arbeiten in der Linguistik}\\
	\textbf{(Technische Übung)}
}

\subtitle{\LaTeX\ -- Teil 6: Mathematikmodus (für Linguisten)}

\author[aMyP]{
	{\small Antonio Machicao y Priemer}
	\\
	{\footnotesize \url{www.linguistik.hu-berlin.de/staff/amyp}}
	%	\\
	%	{\footnotesize \href{mailto:mapriema@hu-berlin.de}{mapriema@hu-berlin.de}}
}

\institute{Institut für deutsche Sprache und Linguistik}

\date{ }

%\publishers{\textbf{6. linguistischer Methodenworkshop \\ Humboldt-Universität zu Berlin}}

%\hyphenation{nobreak}


%%%%%%%%%%%%%%%%%%%%%%%%%%%%%%%%%%%%%%%%%%%%%%%%%%%%
%%%             Preamble's End                   
%%%%%%%%%%%%%%%%%%%%%%%%%%%%%%%%%%%%%%%%%%%%%%%%%%%%      


%%%%%%%%%%%%%%%%%%%%%%%%%%%%%%%%%%%
%%%%%%%%%%%%%%%%%%%%%%%%%%%%%%%%%%%    
%% Title slide 
\begin{frame}
  \HUtitle
\end{frame}


%% Contents slide
\frame{
\begin{multicols}{2}
	\frametitle{Inhaltsverzeichnis}
	\tableofcontents
		[hideallsubsections]
%		[pausesections]
\end{multicols}
	}


%%%%%%%%%%%%%%%%%%%%%%%%%%%%%%%%%%%%
%%%%%%%%%%%%%%%%%%%%%%%%%%%%%%%%%%%%
%% Extra literature

\nocite{Freitag&MyP15a}
\nocite{Knuth1986}
\nocite{Kopka94a}
\nocite{MyP17c}
\nocite{MyP&Kerkhof16a}
	
%%%%%%%%%%%%%%%%%%%%%%%%%%%%%%%%%%%%
%%%%%%%%%%%%%%%%%%%%%%%%%%%%%%%%%%%%


%%%%%%%%%%%%%%%%%%%%%%%%%%%%%%%%%%%%
%%%%%%%%%%%%%%%%%%%%%%%%%%%%%%%%%%%%
%%% Basic literature for these slides

\begin{frame}
\frametitle{Grundlage \& empfohlene Lektüre}

\dots basierend auf \citet{Freitag&MyP15a} und auf \citet{MyP&Kerkhof16a}\\
\ras \href{https://www.researchgate.net/publication/279514740_LATEX-Einfuhrung_fur_Linguisten}{LINK}

\end{frame}


%%%%%%%%%%%%%%%%%%%%%%%%%%%%%%%%%%
%%%%%%%%%%%%%%%%%%%%%%%%%%%%%%%%%%
\section{Einführendes}
\frame{
	\frametitle{~}
	\begin{multicols}{2}
		\tableofcontents[
			currentsection,
			hideallsubsections
			]
	\end{multicols}
}
%%%%%%%%%%%%%%%%%%%%%%%%%%%%%%%%%%

%%%%%%%%%%%%%%%%%%%%%%%%%%%%%%%%%%
\begin{frame}[fragile]
\frametitle{Einführendes}


\begin{itemize}
	\item Im Mathematikmodus werden alle \textbf{Leerzeichen} und \textbf{Zeilenumbrüche} ignoriert und der Text wird \textbf{kursiv} gesetzt.
	
	\item Der Mathematikmodus ist für \textbf{Formeln} gedacht und \textbf{nicht für Text}.
	
	\item Mit dem Befehl \textbf{\ltxterm{textrm}} kann Text mit Leerzeichen und nicht kursiv im Mathematikmodus eingegeben werden. 

\end{itemize}

\pause 
{\small 
\begin{lstlisting}
$Das ist Text im Mathematikmodus$ 

$Das ist \textrm{Text im Mathe-Modus } in textrm eingebettet$
\end{lstlisting}
}

\ea $Das ist Text im Mathematikmodus$ 

\ex $Das ist \textrm{Text im Mathe-Modus } in textrm eingebettet$
\z 

\end{frame}


%%%%%%%%%%%%%%%%%%%%%%%%%%%%%%%%%%

\begin{frame}[fragile]
%\frametitle{Einführendes}


\begin{itemize}
	\item Mathematik-Pakete der American Mathematical Society (AMS)
	
\end{itemize}

\begin{lstlisting}
\usepackage{amsmath}
\usepackage{amsfonts}
\usepackage{amssymb}
\usepackage{amstext}
\usepackage{mathrsfs}
\end{lstlisting}

\end{frame}


%%%%%%%%%%%%%%%%%%%%%%%%%%%%%%%%%%
%%%%%%%%%%%%%%%%%%%%%%%%%%%%%%%%%%
\section{Mathematik-Umgebungen}
\frame{
	\frametitle{~}
	\begin{multicols}{2}
		\tableofcontents[currentsection,hideallsubsections]
	\end{multicols}
}
%%%%%%%%%%%%%%%%%%%%%%%%%%%%%%%%%%
%%%%%%%%%%%%%%%%%%%%%%%%%%%%%%%%%%

\begin{frame}[fragile]
\frametitle{Mathematik-Umgebungen}


Mathematische Ausdrücke können in zwei Varianten geschrieben werden:

\begin{itemize}
	\item in der \textbf{inline}-Variante, umklammert durch \textbf{Dollar-Zeichen} \$:
\end{itemize}

\begin{lstlisting}
Wenn $2^2+\sqrt{2}=c^4$, wie viel beträgt $c$?
\end{lstlisting}
 
\ea Wenn $2^2+\sqrt{2}=c^4$, wie viel beträgt $c$?
\z 

\pause 

\begin{itemize}
	\item im \textbf{Display}-Stil (\emph{Mathematik-Umgebung} im engeren Sinne), eingeschlossen
in einer Kombination aus \textbf{Backslash und eckigen Klammern} \lstinline|\[...\]|:
\end{itemize}

\begin{lstlisting}
Wenn \[2^2+\sqrt{2}=c^4\], wie viel beträgt $c$?
\end{lstlisting}

\ea Wenn \[2^2+\sqrt{2}=c^4\], wie viel beträgt $c$?
\z 

\end{frame}


%%%%%%%%%%%%%%%%%%%%%%%%%%%%%%%%%%
\begin{frame}[fragile]
%\frametitle{Mathematik-Umgebungen}


\begin{itemize}
	\item Für nummerierte Gleichungen: \ltxterm{equation}-Umgebung
\end{itemize}

\begin{lstlisting}
\begin{equation}
\label{eq:FirstEq}
\lim_{n \to \infty}
\sum_{k=1}^n \frac{1}{k^2}
= \frac{\pi^2}{6}
\end{equation}
\end{lstlisting}

\begin{equation}
\label{eq:FirstEq}
	\lim_{n \to \infty}
	\sum_{k=1}^n \frac{1}{k^2}
	= \frac{\pi^2}{6}
\end{equation}

\pause 

\begin{itemize}
	\item Mit \ltxterm{eqref} kann darauf verwiesen werden.
\end{itemize}

\begin{lstlisting}
s. \eqref{eq:FirstEq}; vgl. \ref{eq:FirstEq}
\end{lstlisting}

s. \eqref{eq:FirstEq}; vgl. \ref{eq:FirstEq}


\end{frame}


%%%%%%%%%%%%%%%%%%%%%%%%%%%%%%%%%%
%%%%%%%%%%%%%%%%%%%%%%%%%%%%%%%%%%
\section{Zeichen}
\frame{
	\frametitle{~}
	\begin{multicols}{2}
		\tableofcontents[currentsection,hideallsubsections]
	\end{multicols}
}
%%%%%%%%%%%%%%%%%%%%%%%%%%%%%%%%%%
%%%%%%%%%%%%%%%%%%%%%%%%%%%%%%%%%%

\begin{frame}[fragile]
\frametitle{Zeichen}

\begin{itemize}
	\item \textbf{Viele Zeichen} -- \zB die griechischen Buchstaben \ltxterm{alpha} (\ltxterm{$\alpha$}), \ltxterm{beta} (\ltxterm{$\beta$}), usw.\ -- können \textbf{nur im Mathematikmodus} verwendet werden. 
	
	\item Die Verwendung dieser Zeichen außerhalb des Mathematikmodus \textbf{verhindert die Kompilierung} des Dokuments!

\end{itemize}

\begin{lstlisting}
$\alpha \beta \delta \lambda$
\end{lstlisting}

\ea $\alpha \beta \delta \lambda$
\z 

\end{frame}


%%%%%%%%%%%%%%%%%%%%%%%%%%%%%%%%%%
\begin{frame}\frametitle{(Einige) Zeichen im Mathematikmodus}

\begin{table}
\caption{Allgemeine Zeichen}
\centering
\scalebox{.9}{
\begin{tabular}{ll|ll|ll}
$=$ & \texttt{=} & $\sim$ & \texttt{\textbackslash sim} & $\infty$ & \texttt{\textbackslash infty} \\
$\pm$	&	\texttt{\textbackslash pm}	&$\approx$	&	\texttt{\textbackslash approx}	&$\emptyset$	&	\texttt{\textbackslash emptyset}	\\
$\cdot$	&	\texttt{\textbackslash cdot}	&$\subset$	&	\texttt{\textbackslash subset}	&$\Box$	&	\texttt{\textbackslash Box}	\\
$\times$	&	\texttt{\textbackslash times}	&$\supset$	&	\texttt{\textbackslash supset}	&$\%$	&	\texttt{\textbackslash \%}	\\
$\circ$	&	\texttt{\textbackslash circ}	&$\subseteq$	&	\texttt{\textbackslash subseteq}	&$\$$	&	\texttt{\textbackslash $\$$}	\\
$\in$	&	\texttt{\textbackslash in}	&$\cap$	&	\texttt{\textbackslash cap}		&$\&$	&	\texttt{\textbackslash $\&$}	\\
$\ni$	&	\texttt{\textbackslash ni}	&$\cup$	&	\texttt{\textbackslash cup}		&$\#$	&	\texttt{\textbackslash $\#$}	\\
$\neq$	&	\texttt{\textbackslash neq}	&$\forall$	&	\texttt{\textbackslash forall}	&$\backslash$	&	\texttt{\textbackslash backslash}	\\
$\leq$	&	\texttt{\textbackslash leq}	&$\exists$	&	\texttt{\textbackslash exists}	&$\dots$		&	\texttt{\textbackslash dots}	\\
$\geq$	&	\texttt{\textbackslash geq}	&$\land$	&	\texttt{\textbackslash land}	&$<$	&	\texttt{$<$}	\\
$\ll$	&	\texttt{\textbackslash ll}	&$\lor$		&	\texttt{\textbackslash lor}	&$>$	&	\texttt{$>$}	\\
$\gg$	&	\texttt{\textbackslash gg}	&$\lnot$	&	\texttt{\textbackslash lnot}	&	&	\\
\end{tabular}
}
\end{table}
 \hfill \dots\ keine exhaustive Liste
\end{frame}


%%%%%%%%%%%%%%%%%%%%%%%%%%%%%%%%%%
\begin{frame}\frametitle{(Einige) Zeichen im Mathematikmodus}

\begin{table}
\caption{(Einige) Pfeile, Klammern, Schriften}
\centering
\scalebox{.8}{
\begin{tabular}{llllll}
$\rightarrow$	&	\texttt{\textbackslash rightarrow}	&$\Downarrow$	&	\texttt{\textbackslash Downarrow}	&$\{\}$	&	\texttt{\textbackslash\{\textbackslash\}}	\\
$\leftarrow$	&	\texttt{\textbackslash leftarrow}	&$\mapsto$	&	\texttt{\textbackslash mapsto}	&$\mathcal{A}$	&	\texttt{\textbackslash mathcal\{A\}}	\\
$\leftrightarrow$	&	\texttt{\textbackslash leftrightarrow}	&$\leadsto$	&	\texttt{\textbackslash leadsto}	&$\mathfrak{A}$	&	\texttt{\textbackslash mathfrak\{A\}}	\\
$\Rightarrow$	&	\texttt{\textbackslash Rightarrow}	&$\xrightarrow[abc]{xyz}$	&	\texttt{\textbackslash xrightarrow[abc]\{xyz\}}	&$\mathbb{R}$	&	\texttt{\textbackslash mathbb\{R\}}	\\
$\Leftarrow$	&	\texttt{\textbackslash Leftarrow}	&$()$	&	\texttt{()}	&$\aleph$	&	\texttt{\textbackslash aleph}	\\
$\Leftrightarrow$	&	\texttt{\textbackslash Leftrightarrow}	&$[]$	&\texttt{[]}	&	&	\\
\end{tabular}
}
\end{table}

 \hfill \dots\ keine exhaustive Liste
 
\end{frame}


%%%%%%%%%%%%%%%%%%%%%%%%%%%%%%%%%%
\begin{frame}\frametitle{(Einige) Zeichen im Mathematikmodus}

\begin{table}
\caption{(Einige) griechische Buchstaben}
\centering
%\scalebox{.8}{
\begin{tabular}{ll|ll|ll}
$\alpha$ &	\texttt{\textbackslash alpha}	&$\theta$	&	\texttt{\textbackslash theta}	&$\varepsilon$	&	\texttt{\textbackslash varepsilon}	\\
$\gamma$&	\texttt{\textbackslash gamma}	&$\phi$	&	\texttt{\textbackslash phi}	&$\vartheta$	&	\texttt{\textbackslash vartheta}	\\
$\delta$&	\texttt{\textbackslash delta}	&$\Gamma$	&	\texttt{\textbackslash Gamma}	&$\Phi$	&	\texttt{\textbackslash Phi}	\\
$\epsilon$&	\texttt{\textbackslash epsilon}	&$\Delta$	&	\texttt{\textbackslash Delta}	&$\varphi$	&	\texttt{\textbackslash varphi}	\\
\end{tabular}
%}
\end{table}

 \hfill \dots\ keine exhaustive Liste
 
\end{frame}


%%%%%%%%%%%%%%%%%%%%%%%%%%%%%%%%%%
\begin{frame}\frametitle{(Einige) Zeichen im Mathematikmodus}

\begin{table}
\caption{(Einige) Symbole oberhalb von Zeichen}
\centering
%\scalebox{.8}{
\begin{tabular}{ll|ll|ll}

$\tilde{a}$&	\texttt{\textbackslash tilde\{a\}}	&$\notin$	&	\texttt{\textbackslash notin}	&$\widetilde{abc}$	&	\texttt{\textbackslash widetilde\{abc\}}	\\
$\bar{a}$ &	\texttt{\textbackslash bar\{a\}}	&$\dot{a}$	&	\texttt{\textbackslash dot\{a\}}	&$\overline{abc}$	&	\texttt{\textbackslash overline\{abc\}}	\\
$\vec{a}$&	\texttt{\textbackslash vec\{a\}}	&$\ddot{a}$	&	\texttt{\textbackslash ddot\{a\}}	&$\overrightarrow{abc}$	&	\texttt{\textbackslash overrightarrow\{abc\}}	\\
$\hat{a}$&	\texttt{\textbackslash hat\{a\}}	&$\dot{=}$	&	\texttt{\textbackslash ddot\{=\}}	&$\widehat{abc}$	&	\texttt{\textbackslash widehat\{\}}
\end{tabular}
%} 
\end{table}

 \hfill \dots\ keine exhaustive Liste
 
\end{frame}


%%%%%%%%%%%%%%%%%%%%%%%%%%%%%%%%%%
\begin{frame}

Auflistungen von logischen, mathematischen, u.\,Ä. Zeichen für \LaTeX :

\begin{itemize}
	\item List of logic symbols (Wikipedia): 
	
	\url{https://en.wikipedia.org/wiki/List_of_logic_symbols}
	
	\item \LaTeX\ for Logicians:
	
	\url{http://www.logicmatters.net/latex-for-logicians/}
	
	\item The Great, Big List of \LaTeX\ Symbols: \citet{Carlisle&Co01a}
	
	\item The Comprehensive \LaTeX\ Symbol List -- Symbols accessible from \LaTeX :  \citet{Pakin17a}
\end{itemize}

Zeichnen Sie das benötigte Zeichen und Sie erhalten den Code:

\begin{itemize}
	\item \url{http://detexify.kirelabs.org}
\end{itemize}
\end{frame}


%%%%%%%%%%%%%%%%%%%%%%%%%%%%%%%%%%
%%%%%%%%%%%%%%%%%%%%%%%%%%%%%%%%%%
\section{Mengentheoretische Zeichen}
\frame{
	\frametitle{~}
	\begin{multicols}{2}
		\tableofcontents[currentsection,hideallsubsections]
	\end{multicols}
}
%%%%%%%%%%%%%%%%%%%%%%%%%%%%%%%%%%
%%%%%%%%%%%%%%%%%%%%%%%%%%%%%%%%%%

\begin{frame}[fragile]
\frametitle{Mengentheoretische Zeichen}


{\small
\begin{lstlisting}
$\{\textrm{a}\} \subset \{\textrm{a, e}\}$
\end{lstlisting}
}

\ea $\{\textrm{a}\} \subset \{\textrm{a, e}\}$
\z 


{\small
\begin{lstlisting}
$\emptyset \subseteq \{\textrm{a, b}\}$
\end{lstlisting}
}

\ea $\emptyset \subseteq \{\textrm{a, b}\}$
\z 


{\small
\begin{lstlisting}
$\# \{\emptyset, \textrm{a} \} = 2$
\end{lstlisting}
}

\ea $\# \{\emptyset, \textrm{a} \} = 2$
\z 


{\small
\begin{lstlisting}
$\emptyset \in \{\emptyset, \textrm{a} \}$
\end{lstlisting}
}

\ea $\emptyset \in \{\emptyset, \textrm{a} \}$
\z 
\end{frame}


%%%%%%%%%%%%%%%%%%%%%%%%%%%%%%%%%%
\begin{frame}[fragile]

{\small
\begin{lstlisting}
$\emptyset \notin \{\textrm{a}\}$
\end{lstlisting}
}

\ea $\emptyset \notin \{\textrm{a}\}$
\z 


{\small
\begin{lstlisting}
Wenn $|\textrm{A}| = n$ dann $|\mathfrak{P}(\textrm{A})|=2^{n}$
\end{lstlisting}
}

\ea Wenn $|\textrm{A}| = n$ dann $|\mathfrak{P}(\textrm{A})|=2^{n}$
\z 


{\small
\begin{lstlisting}
$\{\textrm{a, e}\} \setminus \{\textrm{e, u}\} = \{\textrm{a}\}$
\end{lstlisting}
}

\ea $\{\textrm{a, e}\} \setminus \{\textrm{e, u}\} = \{\textrm{a}\}$
\z 

{\small
\begin{lstlisting}
$ \overline{[ \textrm{A} \cup \textrm{B} ]} = 
[ \overline{\textrm{A}} \cap \overline{\textrm{B}} ] $
\end{lstlisting}
}

\ea DeMorgan
$ \overline{[ \textrm{A} \cup \textrm{B} ]} = 
[ \overline{\textrm{A}} \cap \overline{\textrm{B}} ] $
\z 


\end{frame}


%%%%%%%%%%%%%%%%%%%%%%%%%%%%%%%%%%
%%%%%%%%%%%%%%%%%%%%%%%%%%%%%%%%%%
\section{Aussagenlogische Konnektoren}
\frame{
	\frametitle{~}
	\begin{multicols}{2}
		\tableofcontents[currentsection,hideallsubsections]
	\end{multicols}
}
%%%%%%%%%%%%%%%%%%%%%%%%%%%%%%%%%%
%%%%%%%%%%%%%%%%%%%%%%%%%%%%%%%%%%

\begin{frame}[fragile]
\frametitle{Aussagenlogische Konnektoren}

{\small
\begin{lstlisting}
DeMorgans Gesetz:
$\lnot (P \lor Q ) \Leftrightarrow 
(\lnot P \wedge \lnot Q)$

Gesetz des Bikonditionalen:
$(P \leftrightarrow P) \Leftrightarrow 
((P \rightarrow Q) \wedge (Q \rightarrow P))$

Logische Konsequenz:
$((p \rightarrow q) \wedge p) \Rightarrow q$
\end{lstlisting}
}

\pause 

\ea DeMorgans Gesetz: 
$\lnot (P \lor Q ) \Leftrightarrow 
(\lnot P \wedge \lnot Q) $

\ex Gesetz des Bikonditionalen: 
$(P \leftrightarrow Q) \Leftrightarrow 
((P \rightarrow Q) \wedge (Q \rightarrow P))$

\ex Logische Konsequenz: 
$((p \rightarrow q) \wedge p) \Rightarrow q$

\z 

\end{frame}


%%%%%%%%%%%%%%%%%%%%%%%%%%%%%%%%%%
%%%%%%%%%%%%%%%%%%%%%%%%%%%%%%%%%%
\section{Quantoren}
\frame{
	\frametitle{~}
	\begin{multicols}{2}
		\tableofcontents[currentsection,hideallsubsections]
	\end{multicols}
}
%%%%%%%%%%%%%%%%%%%%%%%%%%%%%%%%%%
%%%%%%%%%%%%%%%%%%%%%%%%%%%%%%%%%%

\begin{frame}[fragile]
\frametitle{Quantoren}

{\small
\begin{lstlisting}
$\exists x [$\textsc{frau}$(x)$ $\land
$ \textsc{schlafen}$(x)]$

$\forall x [$\textsc{frau}$(x)$ $\rightarrow
$ \textsc{schlafen}$(x)]$
\end{lstlisting}
}


\pause 


\ea \textbf{Existenzquantor:} \emph{Eine Frau schläft.}

\alert{$\exists x [$\textsc{frau}$(x)$ $\land$ \textsc{schlafen}$(x)]$ }

\gq{Es gibt ein $x$, $x$ ist eine Frau und $x$ schläft.}

$\nrightarrow$ Es gibt nur einen Schlafenden.	 


\pause 


\ex \textbf{Allquantor:} \emph{Jede Frau schläft.}

\alert{$\forall x [$\textsc{frau}$(x)$ $\rightarrow$ \textsc{schlafen}$(x)]$}

\gq{Für alle $x$ gilt, wenn $x$ eine Frau ist dann, schläft $x$.}

$\nrightarrow$ Nur Frauen sind Schlafende. 
\z 

\end{frame}


%%%%%%%%%%%%%%%%%%%%%%%%%%%%%%%%%%
%%%%%%%%%%%%%%%%%%%%%%%%%%%%%%%%%%
\section{Bedeutungsklammern}
\frame{
	\frametitle{~}
	\begin{multicols}{2}
		\tableofcontents[currentsection,hideallsubsections]
	\end{multicols}
}
%%%%%%%%%%%%%%%%%%%%%%%%%%%%%%%%%%
%%%%%%%%%%%%%%%%%%%%%%%%%%%%%%%%%%

\begin{frame}[fragile]
\frametitle{Bedeutungsklammern}


\begin{itemize}
	\item Für die Bedeutungsklammern $\lsem \rsem$ wird das Paket \ltxpack{MnSymbol} benötigt.
\end{itemize}

\begin{lstlisting}
\usepackage{MnSymbol} 
\end{lstlisting}

\pause 

\begin{itemize}
	\item Die Bedeutungsklammern können \textbf{nur im Mathematikmodus} benutzt werden.
\end{itemize}

\begin{lstlisting}
$\lsem \alpha \beta \rsem = 
\lsem \beta \rsem (\lsem \alpha \rsem)$
\end{lstlisting}

\ea $\lsem \alpha \beta \rsem = \lsem \beta \rsem (\lsem \alpha \rsem)$
\z 

\end{frame}


%%%%%%%%%%%%%%%%%%%%%%%%%%%%%%%%%%
%%%%%%%%%%%%%%%%%%%%%%%%%%%%%%%%%%
\section{Klammern für Typen \& Grapheme}
\frame{
	\frametitle{~}
	\begin{multicols}{2}
		\tableofcontents[currentsection,hideallsubsections]
	\end{multicols}
}
%%%%%%%%%%%%%%%%%%%%%%%%%%%%%%%%%%
%%%%%%%%%%%%%%%%%%%%%%%%%%%%%%%%%%

\begin{frame}[fragile]
\frametitle{Klammern für Typen \& Grapheme}


Typen und Grapheme werden \textbf{nicht in Größer-als- und Kleiner-als-Zeichen} (s.\ (\ref{ex:Typ1})) gesetzt, sondern in \textbf{spitzen Klammern} (s.\ (\ref{ex:Typ2})).

{\small
\begin{lstlisting}
$< e, t >$
Das Wort \emph{Achtung} enthält den Digraphen $<$ch$>$.
\end{lstlisting}
}

\ea\label{ex:Typ1} 
	\ea $< e, t >$
	\ex Das Wort \emph{Achtung} enthält den Digraphen $<$ch$>$.
	\z 
\z 


{\small
\begin{lstlisting}
$\langle e, t \rangle$
\emph{Achtung} enthält den Digraphen $\langle$ch$\rangle$.
\end{lstlisting}
}

\ea\label{ex:Typ2} 
	\ea $\langle e, t \rangle$
	\ex \emph{Achtung} enthält den Digraphen $\langle$ch$\rangle$.
	\z 
\z 

\end{frame}


%%%%%%%%%%%%%%%%%%%%%%%%%%%%%%%%%%
%%%%%%%%%%%%%%%%%%%%%%%%%%%%%%%%%%
\section{Eigene Befehle definieren}
\frame{
	\frametitle{~}
	\begin{multicols}{2}
		\tableofcontents[currentsection,hideallsubsections]
	\end{multicols}
}
%%%%%%%%%%%%%%%%%%%%%%%%%%%%%%%%%%
%%%%%%%%%%%%%%%%%%%%%%%%%%%%%%%%%%

\begin{frame}[fragile]
\frametitle{Eigene Befehle definieren}

In \LaTeX\ können Sie eigene Befehle definieren, um lange Zeichenketten wie

{\small
\begin{lstlisting}
$\langle e, t \rangle$

$\langle \langle e,t \rangle , \langle \langle e,t \rangle ,t 
\rangle \rangle$
\end{lstlisting}
}

für kurze Zeichenfolgen wie die in (\ref{ex:Typ3}) zu vermeiden.

\ea\label{ex:Typ3} 
	\ea $\langle e, t \rangle$
	
	\ex $\langle \langle e,t \rangle , \langle \langle e,t \rangle ,t \rangle \rangle$
	\z 
	
\z 

\end{frame}


%%%%%%%%%%%%%%%%%%%%%%%%%%%%%%%%%%
\begin{frame}[fragile]
%\frametitle{Eigene Befehle definieren}

Mit dem Befehl \ltxterm{newcommand}  und der folgenden Syntax definieren Sie den Befehl \ltxterm{typem} mit einem Argumenten. 


{\small
\begin{lstlisting}
\newcommand{\typem}[1]{\langle #1 \rangle}
\end{lstlisting}
}


Das Argument des Befehls wird dann in \textbf{spitze Klammern} gesetzt. Den \textbf{Mathe-Modus} markieren Sie extra.

{\small
\begin{lstlisting}
$\typem{e,t}$
$\typem{\typem{e,t},\typem{\typem{e,t},t}}$
\end{lstlisting}
}


\ea 
	\ea $\typem{e,t}$
	\ex $\typem{\typem{e,t},\typem{\typem{e,t},t}}$
	\z 
\z


\begin{itemize}
	\item Der Befehl \ltxterm{typem} kann in andere \ltxterm{typem}-Befehle eingebettet werden \textbf{weil} der Mathematik-Modus den Befehl umschließt (Extra-Angabe).

	\item \ltxterm{typem} steht für \emph{angle brackets + math-mode}
\end{itemize}
 
\end{frame}


%%%%%%%%%%%%%%%%%%%%%%%%%%%%%%%%%%
\begin{frame}[fragile]
%\frametitle{Eigene Befehle definieren}

Für \textbf{Grapheme} kann ein ähnlicher Befehl \ltxterm{ab} definiert werden, welcher das Argument \textbf{nicht in den Mathematik-Modus} setzt (d.\,h. der Text erscheint nicht kursiv, Leerzeichen und Umlaute werden korrekt wiedergegeben).


{\small
\begin{lstlisting}
\newcommand{\ab}[1]{$\langle$#1$\rangle$} 
\end{lstlisting}
}


Das Argument des Befehls wird dann in \textbf{spitze Klammern} gesetzt, aber \textbf{nicht in den Mathematik-Modus}.

{\small
\begin{lstlisting}
\ab{Öl verschütten}
\end{lstlisting}
}


\ea 
	\ea \ab{Öl verschütten} \hfill [mit \ltxterm{ab}]
	\ex $\typem{Öl verschütten}$ \hfill [mit \ltxterm{typem}]
	\z 
\z

\begin{itemize}
	\item Der Befehl \ltxterm{ab} kann \textbf{nicht} in andere \ltxterm{ab}-Befehle eingebettet werden.

	\item \ltxterm{ab} steht für \emph{angle brackets}
\end{itemize}

\end{frame}


%%%%%%%%%%%%%%%%%%%%%%%%%%%%%%%%%%
\begin{frame}[fragile]
%\frametitle{Eigene Befehle definieren}

Es können auch Befehle \textbf{ohne Argumente} (oder mit \textbf{mehr Argumenten}) definiert werden, und als Abkürzungen benutzt werden:

\newcommand{\citegen}[2]{\citeauthor{#1}s #2 (\citeyear{#1})} 

{\small
\begin{lstlisting}
\newcommand{\ra}{$\rightarrow$}
\newcommand{\citegen}[2]{\citeauthor{#1}s #2 (\citeyear{#1})} 

P \ra\ Q
\citegen{Abney87a}{Dissertation} gilt als Meilenstein 
der NP-Syntax.
\end{lstlisting}
}

\pause 

\ea 
	\ea P \ra\ Q
	\ex \citegen{Abney87a}{Dissertation} gilt als Meilenstein in der NP-Syntax.
	\z 
\z 


\end{frame}


%%%%%%%%%%%%%%%%%%%%%%%%%%%%%%%%%%
\begin{frame}[fragile]
%\frametitle{Eigene Befehle definieren}

\textbf{Befehl mit einem Standard-Argument:} Der folgende Befehl ist so definiert, dass er \textbf{3 Argumente} (\lstinline|[3]|) hat. Für \textbf{das erste Argument} (\lstinline|#1|) ist ein \textbf{Standard-Wert} eingegeben (\lstinline|[$^0$]|), d.\,h. wird der Befehl mit nur zwei Argumenten benutzt (s.\ (\ref{ex:2Arg})), dann wird der Standard-Wert als erstes Argument benutzt (\lstinline|#1|). Der Stardard-Wert kann auch \textbf{mit etwas} (s.\ (\ref{ex:3Arg})) oder \textbf{mit nichts} (s.\ (\ref{ex:Empt})) überschrieben werden. Das erste Argument kann also als \textbf{optionales Argument} benutzt werden.

\newcommand{\headxy}[3][$^0$]{[#2P [#3 #2#1]]}

{\small
\begin{lstlisting}
\newcommand{\headxy}[3][$^0$]{[#2P [#3 #2#1]]}

\headxy{X}{Y}
\headxy[$^\alpha$]{A}{B}
\headxy[]{Z}{W}
\end{lstlisting}
}


\ea 
\ea\label{ex:2Arg} \headxy{X}{Y}
\ex\label{ex:3Arg} \headxy[$^\alpha$]{A}{B}
\ex\label{ex:Empt} \headxy[]{Z}{W}
\z 
\z 


\end{frame}

%%%%%%%%%%%%%%%%%%%%%%%%%%%%%%%%%%
%%%%%%%%%%%%%%%%%%%%%%%%%%%%%%%%%%
\section{Typographisches: Kursiv \vs Recte}
\frame{
	\frametitle{~}
	\begin{multicols}{2}
		\tableofcontents[currentsection,hideallsubsections]
	\end{multicols}
}
%%%%%%%%%%%%%%%%%%%%%%%%%%%%%%%%%%
%%%%%%%%%%%%%%%%%%%%%%%%%%%%%%%%%%

\begin{frame}[fragile]
\frametitle{Typographisches: Kursiv \vs Recte}

\begin{columns}
%%
%%
\begin{column}[t]{.5\textwidth}
%%
\begin{itemize}
	\item Objektsprache: kursiv
	\item Metasprache: recte
	\item Variablen: kursiv
	\item Prädikate (nicht Variable): recte
%	\item Indizes (nicht Variable): recte	
%	\item Mengen (nicht Variable): recte
%	\item Typen: kursiv
\end{itemize}	

\end{column}
%%
%%
\begin{column}[t]{.5\textwidth}

\begin{itemize}
%	\item Objektsprache: kursiv
%	\item Metasprache: recte
%	\item Variablen: kursiv
%	\item Prädikate (nicht Variable): recte
	\item Indizes (nicht Variable): recte	
	\item Mengen (nicht Variable): recte
	\item Typen: kursiv
\end{itemize}	

\end{column}
%%
%%
\end{columns}

\pause

\ea $\lsem [_{\textrm{PP}} in$ $Berlin ] \rsem (s) = \lambda P \lambda x [P(x) \land [x \textrm{ ist in Berlin in } s]]$

	\ea \emph{in Berlin}: Objektsprache
	\ex \emph{s, x, P}: Variablen
	\ex \textrm{ist in Berlin}: invariables Prädikat
	\ex \textrm{PP}: Index
	\z 
\z 

\pause 

{\footnotesize
\begin{lstlisting}
$\lsem [_{\textrm{PP}} in$ $Berlin ] \rsem (s) = \lambda P \lambda x 
[P(x) \land [x \textrm{ ist in Berlin in } s]]$
\end{lstlisting}
}

\end{frame}


%%%%%%%%%%%%%%%%%%%%%%%%%%%%%%%%%%
\begin{frame}[fragile]
%\frametitle{Kursiv \vs Recte}

\begin{columns}
	%%
	%%
	\begin{column}[t]{.5\textwidth}
		%%
		\begin{itemize}
			\item Objektsprache: kursiv
			\item Metasprache: recte
			\item Variablen: kursiv
			\item Prädikate (nicht Variable): recte
			%	\item Indizes (nicht Variable): recte	
			%	\item Mengen (nicht Variable): recte
			%	\item Typen: kursiv
		\end{itemize}	
		
	\end{column}
	%%
	%%
	\begin{column}[t]{.5\textwidth}
		
		\begin{itemize}
			%	\item Objektsprache: kursiv
			%	\item Metasprache: recte
			%	\item Variablen: kursiv
			%	\item Prädikate (nicht Variable): recte
			\item Indizes (nicht Variable): recte	
			\item Mengen (nicht Variable): recte
			\item Typen: kursiv
		\end{itemize}	
		
	\end{column}
	%%
	%%
\end{columns}


\ea $P \in \textrm{D}_{\typem{e,t}}$ und $x \in \textrm{D}_{\typem{e}}$
	\ea \textrm{D}: Menge
	\ex $\typem{e,t}$, $\typem{e}$: Typen
	\z 
\z 

\pause 

{\footnotesize
\begin{lstlisting}
$P \in \textrm{D}_{\typem{e,t}}$ und $x \in \textrm{D}_{\typem{e}}$
\end{lstlisting}
}

\end{frame}


%%%%%%%%%%%%%%%%%%%%%%%%%%%%%%%%%%%%%%%%%%%%%%%%%%%%%%%%%
%%%%%%%%%%%%%%%%%%%%%%%%%%%%%%%%%%%%%%%%%%%%%%%%%%%%%%%%

\section{Hausaufgabe}
\frame{
\begin{multicols}{2}
%\frametitle{~}
	\tableofcontents[currentsection,hideallsubsections]
\end{multicols}
}
%
%%%%%%%%%%%%%%%%%%%%%%%%%%%%%%%%%%%%%%%%%%%%%%%%%%%%%
\begin{frame}{Hausaufgabe 1}

\begin{itemize}
	
	\item Laden Sie folgende Datei aus dem Moodlekurs herunter:
	
	\begin{enumerate}
		\item \ltxterm{test5PDF.pdf}
	\end{enumerate}
		
\end{itemize}

\end{frame}


%%%%%%%%%%%%%%%%%%%%%%%%%%%%%%%%%%%%%%%%%%%%%%%%%%%%
\begin{frame}{Hausaufgabe 2}

\begin{itemize}
	
	\item Installieren Sie die benötigten Pakete in Ihrem \gqq{\texttt{myName.tex}}-Dokument (mit dem Befehl \ltxterm{usepackage}).
	
\end{itemize}

\end{frame}


%%%%%%%%%%%%%%%%%%%%%%%%%%%%%%%%%%%%%%%%%%%%%%%%%%%%%
\begin{frame}{Hausaufgabe 3}

\begin{itemize}
	
	\item Verwenden Sie Ihre \gqq{\texttt{myName.tex}}-Datei vom letzten Mal und
	
	\item geben Sie den benötigten Code ein, um das Ergebnis zu erhalten, das Sie in \gqq{\texttt{test5PDF.pdf}} sehen.
	
	\item Laden Sie dann Ihre \gqq{\texttt{myName.tex}}-Datei und Ihr PDF-Ergebnis bei Moodle hoch. 
	
	(Sie müssen nun 2 Dateien hochladen!)
	
\end{itemize}

\end{frame}


%%%%%%%%%%%%%%%%%%%%%%%%%%%%%%%%%%%%%%%%%%%%%%%%%%%%%
\begin{frame}{Hausaufgabe -- Hinweise}

\begin{itemize}
	
	\item Es gibt einen YouTube-Channel mit \LaTeX -Tutorials:
	
	 \url{https://www.youtube.com/channel/UCC-3dzj6dfbWwGzQzhkUS5A}
	
	\item Bei Twitter finden Sie tägliche \LaTeX -Tweets unter:
	
	\url{https://twitter.com/textip}
	
\end{itemize}

\end{frame}



%%%%%%%%%%%%%%%%%%%%%%%%%%%%%%%%%%%%%%%%%%%%%%%%%%%%%%%%%%
%%%%%%%%%%%%%%%%%%%%%%%%%%%%%%%%%%%%%%%%%%%%%%%%%%%%%%%%%

%\section{XY}
%%\frame{
%%\begin{multicols}{2}
%%\frametitle{~}
%%	\tableofcontents[currentsection]
%%\end{multicols}
%%}
%%
%%%%%%%%%%%%%%%%%%%%%%%%%%%%%%%%%%%%%%%%%%%%%%%%%%%%%
%
%\begin{frame}{XY}
%
%\begin{itemize}
%	\item XY
%\end{itemize}
%
%\end{frame}


%%%%%%%%%%%%%%%%%%%%%%%%%%%%%%%%%%%%%%%%%%%%%%%%%%%%%%%%%%
%%%%%%%%%%%%%%%%%%%%%%%%%%%%%%%%%%%%%%%%%%%%%%%%%%%%%%%%%

%\section{XY}
%%\frame{
%%\begin{multicols}{2}
%%\frametitle{~}
%%	\tableofcontents[currentsection]
%%\end{multicols}
%%}
%%
%%%%%%%%%%%%%%%%%%%%%%%%%%%%%%%%%%%%%%%%%%%%%%%%%%%%%
%
%\begin{frame}{XY}
%
%\begin{itemize}
%	\item XY
%\end{itemize}
%
%\end{frame}


%%%%%%%%%%%%%%%%%%%%%%%%%%%%%%%%%%%%%%%%%%%%%%%%%%%%%%%%%%
%%%%%%%%%%%%%%%%%%%%%%%%%%%%%%%%%%%%%%%%%%%%%%%%%%%%%%%%%

%\section{XY}
%%\frame{
%%\begin{multicols}{2}
%%\frametitle{~}
%%	\tableofcontents[currentsection]
%%\end{multicols}
%%}
%%
%%%%%%%%%%%%%%%%%%%%%%%%%%%%%%%%%%%%%%%%%%%%%%%%%%%%%
%
%\begin{frame}{XY}
%
%\begin{itemize}
%	\item XY
%\end{itemize}
%
%\end{frame}


%%%%%%%%%%%%%%%%%%%%%%%%%%%%%%%%%%%%%%%%%%%%%%%%%%%%%%%%%%
%%%%%%%%%%%%%%%%%%%%%%%%%%%%%%%%%%%%%%%%%%%%%%%%%%%%%%%%%

%\section{XY}
%%\frame{
%%\begin{multicols}{2}
%%\frametitle{~}
%%	\tableofcontents[currentsection]
%%\end{multicols}
%%}
%%
%%%%%%%%%%%%%%%%%%%%%%%%%%%%%%%%%%%%%%%%%%%%%%%%%%%%%
%
%\begin{frame}{XY}
%
%\begin{itemize}
%	\item XY
%\end{itemize}
%
%\end{frame}


%%%%%%%%%%%%%%%%%%%%%%%%%%%%%%%%%%%%%%%%%%%%%%%%%%%%%%%%%%
%%%%%%%%%%%%%%%%%%%%%%%%%%%%%%%%%%%%%%%%%%%%%%%%%%%%%%%%%

%\section{XY}
%%\frame{
%%\begin{multicols}{2}
%%\frametitle{~}
%%	\tableofcontents[currentsection]
%%\end{multicols}
%%}
%%
%%%%%%%%%%%%%%%%%%%%%%%%%%%%%%%%%%%%%%%%%%%%%%%%%%%%%
%
%\begin{frame}{XY}
%
%\begin{itemize}
%	\item XY
%\end{itemize}
%
%\end{frame}