%%%%%%%%%%%%%%%%%%%%%%%%%%%%%%%%%%%%%%%%%%%%%%%%
%% Compile the master file!
%% 		Slides: Antonio Machicao y Priemer
%% 		Course: Wissenschaftliches Arbeiten
%%%%%%%%%%%%%%%%%%%%%%%%%%%%%%%%%%%%%%%%%%%%%%%%


%%%%%%%%%%%%%%%%%%%%%%%%%%%%%%%%%%%%%%%%%%%%%%%%%%%%
%%%             Metadata                         
%%%%%%%%%%%%%%%%%%%%%%%%%%%%%%%%%%%%%%%%%%%%%%%%%%%%  

\title{
	\LaTeX\ for Linguists
}

\subtitle{\LaTeX\ 2: Math mode \& new commands}

\author[aMyP]{
	{\small Sebastian Nordhoff \& Antonio Machicao y Priemer}
	\\
	{\footnotesize \url{www.linguistik.hu-berlin.de/staff/amyp}}
	%	\\
	%	{\footnotesize \href{mailto:mapriema@hu-berlin.de}{mapriema@hu-berlin.de}}
}

\institute{LOT 2019, Amsterdam}

%\date{ }

%\publishers{\textbf{6. linguistischer Methodenworkshop \\ Humboldt-Universität zu Berlin}}

%\hyphenation{nobreak}


%%%%%%%%%%%%%%%%%%%%%%%%%%%%%%%%%%%%%%%%%%%%%%%%%%%%
%%%             Preamble's End                   
%%%%%%%%%%%%%%%%%%%%%%%%%%%%%%%%%%%%%%%%%%%%%%%%%%%%      


%%%%%%%%%%%%%%%%%%%%%%%%%%%%%%%%%%%
%%%%%%%%%%%%%%%%%%%%%%%%%%%%%%%%%%%    
%% Title slide 
\begin{frame}
  \HUtitle
\end{frame}


%% Contents slide
\frame{
\begin{multicols}{2}
	\frametitle{Contents}
%	\tableofcontents[hideallsubsections]
	\tableofcontents
	%[pausesections]
\end{multicols}
	}


%%%%%%%%%%%%%%%%%%%%%%%%%%%%%%%%%%%%
%%%%%%%%%%%%%%%%%%%%%%%%%%%%%%%%%%%%
%% Extra literature

\nocite{Freitag&MyP15a}
\nocite{Knuth1986}
\nocite{Kopka94a}
\nocite{MyP17c}
\nocite{MyP&Kerkhof16a}
	
%%%%%%%%%%%%%%%%%%%%%%%%%%%%%%%%%%%%
%%%%%%%%%%%%%%%%%%%%%%%%%%%%%%%%%%%%


%%%%%%%%%%%%%%%%%%%%%%%%%%%%%%%%%%%%%
%%%%%%%%%%%%%%%%%%%%%%%%%%%%%%%%%%%%%
%%%% Basic literature for these slides
%
%\begin{frame}
%\frametitle{Grundlage \& empfohlene Lektüre}
%
%\dots basierend auf \citet{Freitag&MyP15a} und auf \citet{MyP&Kerkhof16a}\\
%\ras \href{https://www.researchgate.net/publication/279514740_LATEX-Einfuhrung_fur_Linguisten}{LINK}
%
%
%\nocite{Kopka94a}
%
%\end{frame}


%%%%%%%%%%%%%%%%%%%%%%%%%%%%%%%%%%
%%%%%%%%%%%%%%%%%%%%%%%%%%%%%%%%%%
\section{Math mode 1}
\frame{
	\begin{multicols}{2}
		\frametitle{~}
		\tableofcontents[currentsection,hideallsubsections]
	\end{multicols}
}
%%%%%%%%%%%%%%%%%%%%%%%%%%%%%%%%%%

\begin{frame}[fragile]
\frametitle{Math mode 1}

\begin{itemize}
	\item \LaTeX\ has a special mode for \textbf{formulae}. 
	
	\item Text is in \textbf{italics}, \textbf{blanks} and \textbf{line breaks} are \textbf{ignored}.  	
%	\item Der Mathematikmodus ist für \textbf{Formeln} gedacht und \textbf{nicht für Text}.
	
	\item With the command \lstinline|\textrm{ }| inside the math mode, text in upright mode with blanks can be used.
	
\end{itemize}

\pause 
{\small 
\begin{lstlisting}
$You shouldn't use text in math mode.$ 

$You shouldn't use \textrm{ text in math } mode.$
\end{lstlisting}
}

\outputbox{
$You shouldn't use text in math mode.$ 

$You shouldn't use \textrm{ text in math } mode.$
}

\end{frame}


%%%%%%%%%%%%%%%%%%%%%%%%%%%%%%%%%%
%%%%%%%%%%%%%%%%%%%%%%%%%%%%%%%%%%
\subsection{Math environments}
%\frame{
%	\begin{multicols}{2}
%		\frametitle{~}
%		\tableofcontents[currentsection,hideallsubsections]
%	\end{multicols}
%}
%%%%%%%%%%%%%%%%%%%%%%%%%%%%%%%%%%
\begin{frame}[fragile]
\frametitle{Math environments}

Two different math environments can be used for the math mode:

\begin{itemize}
	\item for \textbf{inline} formulae: \lstinline|$ test test $|
\end{itemize}

\begin{lstlisting}
If $2^2+\sqrt{2}=c^4$, what is the value of $c$?
\end{lstlisting}

\outputbox{
%\ea 
If $2^2+\sqrt{2}=c^4$, what is the value of $c$?
%\z 
}
\pause 

\begin{itemize}
	\item \textbf{display} style (\emph{math environment} in narrow sense): \lstinline|\[ ... \]|
\end{itemize}

\begin{lstlisting}
If $2^2+\sqrt{2}=c^4$, what is the value of $c$?
\end{lstlisting}

\outputbox{
%\ea 
If \[2^2+\sqrt{2}=c^4\], what is the value of $c$?
%\z 
}

\end{frame}


%%%%%%%%%%%%%%%%%%%%%%%%%%%%%%%%%%
%%%%%%%%%%%%%%%%%%%%%%%%%%%%%%%%%%
\subsection{Equation environment}
%\frame{
%	\begin{multicols}{2}
%		\frametitle{~}
%		\tableofcontents[currentsection,hideallsubsections]
%	\end{multicols}
%}
%%%%%%%%%%%%%%%%%%%%%%%%%%%%%%%%%%
\begin{frame}[fragile]
\frametitle{Equation environment}


%\begin{itemize}
%\item 
For \textbf{numbered equations}: \ltxterm{equation} environment
%\end{itemize}

\begin{multicols}{2}
	
\begin{lstlisting}
\begin{equation}
\label{eq:FirstEq}
\lim_{n \to \infty}
\sum_{k=1}^n \frac{1}{k^2}
= \frac{\pi^2}{6}
\end{equation}
\end{lstlisting}

\columnbreak

\begin{equation}
\label{eq:FirstEq}
\lim_{n \to \infty}
\sum_{k=1}^n \frac{1}{k^2}
= \frac{\pi^2}{6}
\end{equation}

\end{multicols}

\pause 

%\begin{itemize}
%\item 
For \textbf{cross references} to numbered equations \lstinline|\eqref{ }| can be used.
%\end{itemize}

\begin{multicols}{2}
	
\begin{lstlisting}
see \eqref{eq:FirstEq}

see \ref{eq:FirstEq}
\end{lstlisting}

\columnbreak

%\outputbox{
see \eqref{eq:FirstEq}

see \ref{eq:FirstEq}
%}

\end{multicols}

\end{frame}


%%%%%%%%%%%%%%%%%%%%%%%%%%%%%%%%%%
%%%%%%%%%%%%%%%%%%%%%%%%%%%%%%%%%%
\subsection{Math packages}
%\frame{
%	\begin{multicols}{2}
%		\frametitle{~}
%		\tableofcontents[currentsection,hideallsubsections]
%	\end{multicols}
%}
%%%%%%%%%%%%%%%%%%%%%%%%%%%%%%%%%%
\begin{frame}[fragile]
\frametitle{Math packages}

Some symbols can be used only when specific math packages are loaded.

\bigskip

Math packages from the American Mathematical Society (AMS)

\begin{lstlisting}
\usepackage{amsmath}
\usepackage{amsfonts}
\usepackage{amssymb}
\usepackage{amstext}
\usepackage{mathrsfs}
\end{lstlisting}

\end{frame}


%%%%%%%%%%%%%%%%%%%%%%%%%%%%%%%%%%
%%%%%%%%%%%%%%%%%%%%%%%%%%%%%%%%%%
\section{Customizing your commands}
\frame{
	\frametitle{~}
	\begin{multicols}{2}
		\tableofcontents[currentsection,hideallsubsections]
	\end{multicols}
}
%%%%%%%%%%%%%%%%%%%%%%%%%%%%%%%%%%
%%%%%%%%%%%%%%%%%%%%%%%%%%%%%%%%%%

\begin{frame}[fragile]
\frametitle{Customizing your commands}

You can create your own commands!

\begin{multicols}{2}
\begin{lstlisting}
$\langle e, t \rangle$

$\langle \langle e,t \rangle , \langle 
\langle e,t \rangle ,t \rangle \rangle$
\end{lstlisting}

\columnbreak 

$\langle e, t \rangle$

\smallskip

$\langle \langle e,t \rangle , \langle \langle e,t \rangle ,t \rangle \rangle$

\end{multicols}

\pause

%\end{frame}
%
%
%%%%%%%%%%%%%%%%%%%%%%%%%%%%%%%%%%%
%\begin{frame}[fragile]
%%\frametitle{Eigene Befehle definieren}

Defining a command with \textbf{one argument} (for semantic types):

{\small
\begin{lstlisting}
\newcommand{\type}[1]{\langle #1 \rangle}
\end{lstlisting}
}

\smallskip

The argument of the new command will be in angled brackets:

\vspace{-.25cm}

\begin{columns}
	
\column[c]{.55\textwidth}

%\begin{multicols}{2}
{\small
\begin{lstlisting}
$\type{e,t}$

$\type{\typem{e,t},\typem{\typem{e,t},t}}$
\end{lstlisting}
}

%\columnbreak

\column[c]{.4\textwidth}

$\typem{e,t}$

\smallskip

$\typem{\typem{e,t},\typem{\typem{e,t},t}}$

\end{columns}

%\end{multicols}

\bigskip


\lstinline|\typem{ }| can be embedded in further \lstinline|\typem{ }| commands!

\end{frame}


%%%%%%%%%%%%%%%%%%%%%%%%%%%%%%%%%%
\begin{frame}[fragile]
%\frametitle{Eigene Befehle definieren}

Defining a command with \textbf{one argument} (for graphemes):

{\small
\begin{lstlisting}
\newcommand{\ab}[1]{$\langle$#1$\rangle$} 
\end{lstlisting}
}

The argument of the new command will be in angled brackets, but not in math mode:

{\small
\begin{lstlisting}
\ab{buying a house}
\end{lstlisting}
}


\ea 
\ea \ab{buying a house} \hfill [mit \ltxterm{ab}]
\ex $\typem{buying a house}$ \hfill [mit \ltxterm{typem}]
\z 
\z


\lstinline|\ab{ }| cannot embed further \lstinline|\ab{ }| commands!


\end{frame}


%%%%%%%%%%%%%%%%%%%%%%%%%%%%%%%%%%
\begin{frame}[fragile]
%\frametitle{Eigene Befehle definieren}

Defining a command \textbf{without arguments} (for abbreviations):

\begin{multicols}{2}
	
{\small
\begin{lstlisting}
\newcommand{\ra}{$\rightarrow$}

P \ra\ Q
\end{lstlisting}
}

\columnbreak

P \ra\ Q
 
\end{multicols}

\pause

\bigskip 

Defining a command with \textbf{more than one argument}:

{\small
\begin{lstlisting}
\newcommand{\citegen}[3]{#1's #2 (#3)} 

\citegen{Abney}{dissertation}{1987} is considered a milestone in NP Syntax.
\end{lstlisting}
}

\newcommand{\citegen}[3]{#1's #2 (#3)} 

\citegen{Abney}{dissertation}{1987} is considered a milestone in NP Syntax.

\end{frame}


%%%%%%%%%%%%%%%%%%%%%%%%%%%%%%%%%%%
%\begin{frame}[fragile]
%%\frametitle{Eigene Befehle definieren}
%
%\textbf{Befehl mit einem Standard-Argument:} Der folgende Befehl ist so definiert, dass er \textbf{3 Argumente} (\lstinline|[3]|) hat. Für \textbf{das erste Argument} (\lstinline|#1|) ist ein \textbf{Standard-Wert} eingegeben (\lstinline|[$^0$]|), d.\,h. wird der Befehl mit nur zwei Argumenten benutzt (s.\ (\ref{ex:2Arg})), dann wird der Standard-Wert als erstes Argument benutzt (\lstinline|#1|). Der Stardard-Wert kann auch \textbf{mit etwas} (s.\ (\ref{ex:3Arg})) oder \textbf{mit nichts} (s.\ (\ref{ex:Empt})) überschrieben werden. Das erste Argument kann also als \textbf{optionales Argument} benutzt werden.
%
%\newcommand{\headxy}[3][$^0$]{[#2P [#3 #2#1]]}
%
%{\small
%\begin{lstlisting}
%\newcommand{\headxy}[3][$^0$]{[#2P [#3 #2#1]]}
%
%\headxy{X}{Y}
%\headxy[$^\alpha$]{A}{B}
%\headxy[]{Z}{W}
%\end{lstlisting}
%}
%
%
%\ea 
%\ea\label{ex:2Arg} \headxy{X}{Y}
%\ex\label{ex:3Arg} \headxy[$^\alpha$]{A}{B}
%\ex\label{ex:Empt} \headxy[]{Z}{W}
%\z 
%\z 
%
%\end{frame}


%%%%%%%%%%%%%%%%%%%%%%%%%%%%%%%%%
\begin{frame}[fragile]
\frametitle{Exercise}

Go to \url{https://github.com/langsci/latex4linguists/blob/master/1-2.md}\\
and follow the instructions of the \textbf{five blocks} in your \texttt{.tex} file.

\end{frame}


%%%%%%%%%%%%%%%%%%%%%%%%%%%%%%%%%%%
%%%%%%%%%%%%%%%%%%%%%%%%%%%%%%%%%%%
%\section{XY}
%%\frame{
%%\begin{multicols}{2}
%%\frametitle{~}
%%	\tableofcontents[currentsection]
%%\end{multicols}
%%}
%%%%%%%%%%%%%%%%%%%%%%%%%%%%%%%%%%%
%
%\begin{frame}{XY}
%
%\begin{itemize}
%	\item XY
%\end{itemize}
%
%\end{frame}


%%%%%%%%%%%%%%%%%%%%%%%%%%%%%%%%%%%%
%%%%%%%%%%%%%%%%%%%%%%%%%%%%%%%%%%%%
%\iftoggle{handout}{
%	
%%%%%%%%%%%%%%%%%%%%%%%%%%%%%%%%%%%%
%\begin{frame}
%%\frametitle{Bücher \& Artikel}
%	
%Test Toggle ON
%	
%\end{frame}
%
%}
%%% END handout true 
%%% BEGIN handout false
%{
%%%%%%%%%%%%%%%%%%%%%%%%%%%%%%%%%%%
%
%%% EMPTY
%
%}%% END HO-Toggle
%%%%%%%%%%%%%%%%%%%%%%%%%%%%%%%%%%%