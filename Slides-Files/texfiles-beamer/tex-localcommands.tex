%%%%%%%%%%%%%%%%%%%%%%%%%%%%%%%%%%%%%%%%%%%%%%%%%%%%
%%%           MyP-Commands  2018.12.08       
%%%%%%%%%%%%%%%%%%%%%%%%%%%%%%%%%%%%%%%%%%%%%%%%%%%%   


%%%%%%%%%%%%%%%%%%%%%%%%%%%%%%%%
% German quotation marks:
\newcommand{\gqq}[1]{\glqq{}#1\grqq{}}		%double
\newcommand{\gq}[1]{\glq{}#1\grq{}}			%simple


%%%%%%%%%%%%%%%%%%%%%%%%%%%%%%%%
% Abbreviations in German
% package needed: xspace
% Short space in German abbreviations: \,	
\newcommand{\dash}{\mbox{d.\,h.}\xspace}
\newcommand{\idR}{\mbox{i.\,d.\,R.}\xspace}
\newcommand{\su}{\mbox{s.\,u.}\xspace}
\newcommand{\ua}{\mbox{u.\,a.}\xspace}
\newcommand{\va}{\mbox{v.\,a.}\xspace}
\newcommand{\zB}{\mbox{z.\,B.}\xspace}
%\newcommand{\s}{s.~}
%not possibel: \dh --> d.\,h.


%%%%%%%%%%%%%%%%%%%%%%%%%%%%%%%%
%Abbreviations in English
\newcommand{\ao}{a.o.\ }	% among others
\newcommand{\cf}[1]{(cf.~#1)}	% confer = compare
\newcommand{\cfe}[1]{(cf.~(\ref{#1}))}	% compare + example
\newcommand{\ia}{i.a.}	% inter alia = among others
\newcommand{\ie}{i.e.~}	% id est = that is
\newcommand{\fe}{e.g.~}	% exempli gratia = for example
%not possible: \eg --> e.g.~
\newcommand{\vs}{vs.\ }	% versus
\newcommand{\wrt}{w.r.t.\ }	% with respect to


%%%%%%%%%%%%%%%%%%%%%%%%%%%%%%%%
% Dash:
\newcommand{\gs}[1]{--\,#1\,--}


%%%%%%%%%%%%%%%%%%%%%%%%%%%%%%%%
% Rightarrow with and without space
\def\ra{\ensuremath\rightarrow}			%without space
\def\ras{\ensuremath\rightarrow\ }		%with space
\def\la{\ensuremath\leftarrow}
\def\las{\ensuremath\leftarrow\ }


%%%%%%%%%%%%%%%%%%%%%%%%%%%%%%%%
%% X-bar notation

%% Notation with primes (not emphasized): \xprime{X}
\newcommand{\xprime}[1]{#1$^{\prime}$}
\newcommand{\xxprime}[1]{#1$^{\prime\prime}$}
\newcommand{\xxxprime}[1]{#1$^{\prime\prime\prime}$}

%% Notation with primes (emphasized): \exbar{X}
\newcommand{\exprime}[1]{\emph{#1}$^{\prime}$}
\newcommand{\exxprime}[1]{\emph{#1}$^{\prime\prime}$}
\newcommand{\exxxprime}[1]{\emph{#1}$^{\prime\prime\prime}$}

% Notation with zero and max (not emphasized): \xbar{X}
\newcommand{\xzero}[1]{#1$^{0}$}
\newcommand{\maxbar}[1]{#1$^{\textsc{max}}$}

% Notation with zero and max (emphasized): \xbar{X}
\newcommand{\ezerobar}[1]{\emph{#1}$^{0}$}
\newcommand{\emaxbar}[1]{\emph{#1}$^{\textsc{max}}$}

%% Notation with bars (already implemented in gb4e):
% \obar{X}, \ibar{X}, \iibar{X}, \mbar{X} %Problems with \mbar!
% $\overline{}$
%
%% Without gb4e:
\newcommand{\overbar}[1]{\mkern 1.5mu\overline{\mkern-1.5mu#1\mkern-1.5mu}\mkern 1.5mu}
%
%%% OR:
%\newcommand{\ibar}[1]{$\overline{\textrm{#1}}$}
%\newcommand{\iibar}[1]{$\overline{\overline{\textrm{#1}}}$}
%% (emphasized):
\newcommand{\eibar}[1]{$\overline{#1}$}
\newcommand{\eiibar}[1]{\overline{$\overline{#1}}$}


%%%%%%%%%%%%%%%%%%%%%%%%%%%%%%%%
%% Subscript & Superscript: no italics inside math mode
\newcommand{\down}[1]{\textsubscript{#1}}
\newcommand{\downm}[1]{_{\textrm{#1}}}

\newcommand{\up}[1]{\textsuperscript{#1}}
\newcommand{\upm}[1]{^{\textrm{#1}}}


%%%%%%%%%%%%%%%%%%%%%%%%%%%%%%%%
%% Small caps subscripts
\newcommand{\scdown}[1]{\textsubscript{\textsc{#1}}}


%%%%%%%%%%%%%%%%%%%%%%%%%%%%%%%%%
%%% Shorter Underline
%\DeclareTextCommand{\_}{T1}{\leavevmode \kern.06em\vbox{\hrule width.4em}}


%%%%%%%%%%%%%%%%%%%%%%%%%%%%%%%%
%% Object- and Meta-language marking:
%\newcommand{\obj}[1]{\glqq{}#1\grqq{}}		%German double quotes
%\newcommand{\obj}[1]{``#1''}					  %English double quotes
\newcommand{\obj}[1]{\emph{#1}}                 %Emphasising
\newcommand{\term}[1]{\textsc{#1}}              %for abbreviated terminology


%%%%%%%%%%%%%%%%%%%%%%%%%%%%%%%%
% Size:
\newcommand{\size}[1]{{\footnotesize #1}}	% f.e. resize citations


%%%%%%%%%%%%%%%%%%%%%%%%%%%%%%%%
%% for LaTeX terminology: package names, environments, commands
\newcommand{\ltxterm}[1]{{\footnotesize \texttt{#1}}}
\newcommand{\ltxpack}[1]{{\footnotesize \texttt{#1}}}


%%%%%%%%%%%%%%%%%%%%%%%%%%%%%%%%
% Short cuts (<STRG + ALT>):
%\newcommand{\short}[1]{\texttt{\textsc{#1}}}		%Emphasising
\newcommand{\short}[1]{$\langle$\texttt{\textsc{#1}}$\rangle$}		%Emphasising


%%%%%%%%%%%%%%%%%%%%%%%%%%%%%%%%
% Writing text with colour:
% package needed: xcolor
% Command \alert{} in Beamer >> red
\newcommand{\blue}[1]{\textcolor{blue}{#1}}
\newcommand{\green}[1]{\textcolor{green}{#1}}
\newcommand{\red}[1]{\textcolor{red}{#1}}


%%%%%%%%%%%%%%%%%%%%%%%%%%%%%%%%
%% Marking text with colour: 
%%% package needed: color
\newcommand{\clrr}[1]{\colorbox{red}{#1}}
\newcommand{\clry}[1]{\colorbox{yellow}{#1}}


%%%%%%%%%%%%%%%%%%%%%%%%%%%%%%%%
%% Semantic types (<e,t>), features, variables and graphemes in angled brackets 

%%% types and variables, in math mode: angled brackets + italics + no space
%\newcommand{\type}[1]{$<#1>$}

%%% OR more correctly: 
%%% Types and Variables: chevrons! + text in math mode (italics + no space)
\newcommand{\type}[1]{$\langle #1 \rangle$} %% In Math Mode, only single types
\newcommand{\typem}[1]{\langle #1 \rangle } %% Mathmode extra, complex types

%%% Features and Graphemes: chevrons! + normal font
\newcommand{\ab}[1]{$\langle$#1$\rangle$} %% no italics
\newcommand{\abe}[1]{$\langle$\emph{#1}$\rangle$} %% italics


%%%%%%%%%%%%%%%%%%%%%%%%%%%%%%%%
%% Function symbol in Beamer Class!
%% italics and serif
\newcommand{\func}{\emph{\textrm{f}}}
\newcommand{\gunc}{\emph{\textrm{g}}}
\newcommand{\chiF}[1]{\chi _{\textrm{#1}}} 


%%%%%%%%%%%%%%%%%%%%%%%%%%%%%%%%
%% HPSG: Features and Values!
\newcommand{\wert}[1]{\emph{#1}}		%Values & Types
\newcommand{\val}[1]{\emph{#1}}		%Values & Types
\newcommand{\feat}[1]{\textsc{#1}}	%Features


%%%%%%%%%%%%%%%%%%%%%%%%%%%%%%%%
%% (Syntactic) Trees
% package needed: forest
%
%%% Setting for simple trees
%\forestset{
%	sn edges/.style={for tree={parent anchor=south, child anchor=north}}
%}

%%% Setting for complex trees
%\forestset{
%	sn edges/.style={for tree={parent anchor=south, child anchor=north,align=center,base=bottom,where n children=0{tier=word,inner xsep=0pt,outer sep=0pt}{}}}, 
%background tree/.style={for tree={text opacity=0.2,draw opacity=0.2,edge={draw opacity=0.2}}}
%}
%
%\newcommand\HideWd[1]{%
%	\makebox[0pt]{#1}%
%}


%%%%%%%%%%%%%%%%%%%%%%%%%%%%%%%%
%% Outputbox
\newcommand{\outputbox}[1]{\noindent\fbox{\parbox[t][][t]{0.98\linewidth}{#1}}\vspace{0.5em}}


%%%%%%%%%%%%%%%%%%%%%%%%%%%%%%%%
% Margin notes: \myp{NOTE}
% package needed: marginnote
\renewcommand{\marginfont}{\singlespacing}
\renewcommand{\marginfont}{\footnotesize}
\renewcommand{\marginfont}{\color{black}}

\newcommand{\myp}[1]{%
	\marginnote{%
		\begin{spacing}{1}
			\vspace{-\baselineskip}%
			\color{red}\scriptsize#1
		\end{spacing}
	}
}


%%%%%%%%%%%%%%%%%%%%%%%%%%%%%%%%
%% Corpora & Grammmars
\newcommand{\DWDS}[1]{DWDS\nocite{DWDS}: #1}
 
\newcommand{\CREA}{\citetext{CREA \citeyear{CREA}}}
\newcommand{\CREAA}{\citetext{CREA \citeyear{CREAA}}}
\newcommand{\CORPES}{\citetext{CORPES \citeyear{CORPES}}}

\newcommand{\DECOW}{\citetext{DECOW \citeyear{SchaeferR15a}}}
\newcommand{\ESCOW}{\citetext{ESCOW \citeyear{SchaeferR&Co12a}}}

\newcommand{\RAEa}[1]{RAE, \citeyear[#1]{RAE10a}}
\newcommand{\RAEb}[1]{RAE, \citeyear[#1]{RAE10b}}


%%%%%%%%%%%%%%%%%%%%%%%%%%%%%%%%
%% Literature and Appendix
\newcommand{\backupbegin}{
	\newcounter{finalframe}
	\setcounter{finalframe}{\value{framenumber}}
}
\newcommand{\backupend}{
	\setcounter{framenumber}{\value{finalframe}}
}


%%%%%%%%%%%%%%%%%%%%%%%%%%%%%%%%%%%%%%%%%%%%%%%%%%%%
%%%          Useful commands                    
%%%%%%%%%%%%%%%%%%%%%%%%%%%%%%%%%%%%%%%%%%%%%%%%%%%%


%%%%%%%%%%%%%%%%%%%%%%%%%%%%%%%%%%%
%%%%%%%%%%%%%%%%%%%%%%%%%%%%%%%%%%%
%\section{XY}
%%\frame{
%%\begin{multicols}{2}
%%\frametitle{~}
%%	\tableofcontents[currentsection]
%%\end{multicols}
%%}
%%%%%%%%%%%%%%%%%%%%%%%%%%%%%%%%%%%
%
%\begin{frame}{XY}
%
%\begin{itemize}
%	\item XY
%\end{itemize}
%
%\end{frame}


%%%%%%%%%%%%%%%%%%%%%%%%%%%%%%%%%%%
%%%%%%%%%%%%%%%%%%%%%%%%%%%%%%%%%%%
%\iftoggle{handout}{
%	
%%%%%%%%%%%%%%%%%%%%%%%%%%%%%%%%%%%
%\begin{frame}
%%\frametitle{Bücher \& Artikel}
%	
%Test Toggle ON
%	
%\end{frame}
%
%}
%%% END handout true 
%%% BEGIN handout false
%{
%%%%%%%%%%%%%%%%%%%%%%%%%%%%%%%%%%%
%
%%% EMPTY
%
%}%% END HO-Toggle
%%%%%%%%%%%%%%%%%%%%%%%%%%%%%%%%%%%


%%%%%%%%%%%%%%%%%%%%%
%% COMMENTING BLOCKS
% \if0    \fi 

%%%%%%%%%%%%%%%%%%%%%
%% FOR ITEMS:
%\begin{itemize}
%  \item<2-> from point 2
%  \item<3-> from point 3 
%  \item<4-> from point 4 
%\end{itemize}
%
% or: \onslide<2->
% or: \visible<overlay specification>{text}
% or: \only<overlay specification>{text}
% or: \pause

%%%%%%%%%%%%%%%%%%%%%
%% JAMBOX FOR EXAMPLES:
%\ea 
%\settowidth\jamwidth{ Test} 
%Die Studierenden, die weitgehend von Stipendien leben, erhalten einen Mietzuschuss. 
%\jambox{Test}
%\z 

%%%%%%%%%%%%%%%%%%%%%
%% VERTICAL SPACE:
% \vspace{.5cm}
% \vfill

%%%%%%%%%%%%%%%%%%%%%
% RED MARKING OF TEXT:
%\alert{bis spätestens Mittwoch, 18 Uhr}

%%%%%%%%%%%%%%%%%%%%%
%% RESCALE BIG TABLES:
%\scalebox{0.8}{
%For Big Tables
%}

%%%%%%%%%%%%%%%%%%%%%
%% BLOCKS:
%\begin{alertblock}{Title}
%Text
%\end{alertblock}
%
%\begin{block}{Title}
%Text
%\end{block}
%
%\begin{exampleblock}{Title}
%Text
%\end{exampleblock}

%%%%%%%%%%%%%%%%%%%%%
%% MINIPAGE:
%\begin{minipage}[ÄUSSERE POSITION][HÖHE][INNERE POSITION]{BREITE}
%	Beispieltext
%\end{minipage}
%
%% MINIPAGE EXAMPLE:
%\begin{minipage}[b][][c]{.45\textwidth}
%	\onslide<2->
%	\begin{figure}
%		\centering
%		\includegraphics[scale=.1]{../material/Vierecke-Fee1}
%		\caption{Vierecke 1}
%	\end{figure}
%\end{minipage}
%%
%\begin{minipage}[b][][c]{.45\textwidth}
%	\onslide<3->
%	\begin{figure}
%		\centering
%			\includegraphics[scale=.1]{../material/Vierecke-Fee2}
%		\caption{Vierecke 2}
%	\end{figure}
%\end{minipage}	
%%

%%%%%%%%%%%%%%%%%%%%%
%% VIDEO:
%\begin{frame}
%\frametitle{Vor fast 54 Jahren in Berlin \dots}
%
%\begin{center}
%	\includemedia[
%	width=0.9\linewidth,height=0.506\linewidth,
%	activate=pageopen,
%	flashvars={
%		modestbranding=1 % no YT logo in control bar
%		&autohide=1 % controlbar autohide
%		&showinfo=0 % no title and other info before start
%		&rel=0 % no related videos after end
%		&playsinline=0
%		&start=28.5
%	}
%	]{}{http://www.youtube.com/v/NaZ3onbUrew}
%\end{center}
%
%\end{frame}


%%%%%%%%%%%%%%%%%%%%%%%%%%%%%%%%%%%%%%%%%%%%%%%%%%%%
%%%          NOTES: To Check                    
%%%%%%%%%%%%%%%%%%%%%%%%%%%%%%%%%%%%%%%%%%%%%%%%%%%% 

% Do I use ``'' or \emph{•} for object language?
