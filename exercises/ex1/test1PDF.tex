%%%%%%%%%%%%%%%%%%%%%%%%%%%%%%%%%%%%%%%%%%%%%
% Compile: PDFLaTeX BibTeX PDFLaTeXP PDFLaTeX
%%%%%%%%%%%%%%%%%%%%%%%%%%%%%%%%%%%%%%%%%%%%%

\documentclass[10pt,paper=a4,abstracton]{scrartcl}

%%%%%%%%%%%%%PACKAGES%%%%%%%%%%%%%

%\usepackage[utf8]{inputenc}
\usepackage[english,ngerman]{babel}
%\usepackage{blindtext}
\usepackage[T3,T1]{fontenc}
\usepackage{lmodern}
%\usepackage{natbib}
%\usepackage{graphicx}
%\usepackage{booktabs, array}
%\usepackage{linguex}
%\usepackage[noenc,safe]{tipa}
%\usepackage{etex}		%For Forest bug
%\usepackage{forest}
%\usepackage{tikz-qtree}
%\usepackage{xspace}
%\usepackage{setspace}
%\usepackage{listings}
%\usepackage{multicol}
%\usepackage[bookmarksnumbered]{hyperref}


%%%%%%%%%%%%%%META DATA%%%%%%%%%%%%%
%\author{$\langle$Ihr Name$\rangle$ \and $\langle$noch ein Name$\rangle$}
%\title{\LaTeX\ für Linguisten}
%\subtitle{Meine erste \LaTeX -Datei}
%\date{$\langle$Das heutige Datum$\rangle$}



\begin{document}


%%%%%%%%%%%%%META DATA%%%%%%%%%%%%%
\author{$\langle$Ihr Name$\rangle$ \and $\langle$noch ein Name$\rangle$}
\title{\LaTeX\ für Linguisten}
\subtitle{Meine erste \LaTeX -Datei}
\date{$\langle$Das heutige Datum$\rangle$}

\maketitle

\begin{abstract}
	Ein Abstract ist eine kurze Zusammenfassung über den Inhalt 
	der Arbeit. Das Abstract wird immer am Anfang des Dokuments 
	positioniert.\par
	Es ist auch möglich das Abstract in mehrere Absätze
	zu teilen.
\end{abstract}

\tableofcontents


%%%%%%%%%%%%%%%%%%%%%%%%
\section{Hausaufgabe 1}


%%%%%%%%%%%%%%%%%%%%%%%%
\subsection[Zeichen und Sonderzeichen]{Hier werden Zeichen und Sonderzeichen geübt}

Folgende Zeichen können bei \LaTeX\ nicht direkt benutzt werden: \# \$ \& \_ \{ \} \%. Für die folgenden Zeichen braucht man andere Befehle: \textbackslash, \textgreater , \textless , \textasciicircum .


%%%%%%%%%%%%%%%%%%%%%%%%
\subsubsection[Erste Subsection]{Das ist eine Subsection}

Diese Datei ist dazu gemacht, dass alle Workshopteilnehmer einige Vorzüge von \LaTeX\ austesten können.


%%%%%%%%%%%%%%%%%%%%%%%%
\subsubsection[Zweite Subsection]{Das ist eine weitere Subsection}

Diese Datei ist dazu gemacht, dass alle Workshopteilnehmer einige Vorzüge von \LaTeX\ austesten können.


%%%%%%%%%%%%%%%%%%%%%%%%
\subsection[Fußnoten]{Hier werden Fußnoten geübt}

Diese Datei ist dazu gemacht, dass alle Workshopteilnehmer\footnote{Hier ist eine Fußnote.} einige Vorzüge von \LaTeX\ austesten können.\footnote{Hier ist noch eine Fußnote.}


%%%%%%%%%%%%%%%%%%%%%%%%
\subsection[Textauszeichnung]{Hier wird die Textauszeichnung geübt}

Diese Datei ist {\tiny dazu} gemacht, dass {\Huge alle} Workshopteilnehmer einige {\Large Vorzüge} von \LaTeX\ austesten können.

Diese \textbf{Datei} ist dazu \underline{gemacht}, dass alle \textsc{Workshopteilnehmer} einige \emph{Vorzüge} von \LaTeX\ austesten \texttt{können}.

\noindent Diese Datei ist dazu gemacht, dass alle Workshopteilnehmer einige Vorzüge von \LaTeX\ austesten können.


%%%%%%%%%%%%%%%%%%%%%%%%
\section{Hausaufgabe 2}


%%%%%%%%%%%%%%%%%%%%%%%%%
%\subsection{Textumgebungen}
%
%
%%%%%%%%%%%%%%%%%%%%%%%%%
%\subsubsection{Quote und Quotation}
%
% Diese Datei ist dazu gemacht, dass alle Workshopteilnehmer einige Vorzüge von \LaTeX\ austesten können. Das ist der Text vor der \texttt{quote}-Umgebung.
%\begin{quote}
%	Die grammatischen Phänomene in einer Sprache zerfallen in zwei Teilbereiche: kerngrammatische und randgrammatische Phänomene (\emph{Ausnahmen}).
%	
%	Die grammatischen Phänomene in einer Sprache zerfallen in zwei Teilbereiche: kerngrammatische und randgrammatische Phänomene (\emph{Ausnahmen}).
%\end{quote}
%Das ist der Text nach der \texttt{quote}-Umgebung. Diese Datei ist dazu gemacht, dass alle Workshopteilnehmer einige Vorzüge von \LaTeX\ austesten können.
%
%
% Diese Datei ist dazu gemacht, dass alle Workshopteilnehmer einige Vorzüge von \LaTeX\ austesten können. Das ist der Text vor der \texttt{quotation}-Umgebung.
%\begin{quotation}
%	Die grammatischen Phänomene in einer Sprache zerfallen in zwei Teilbereiche: kerngrammatische und randgrammatische Phänomene (\emph{Ausnahmen}).
%	
%	Die grammatischen Phänomene in einer Sprache zerfallen in zwei Teilbereiche: kerngrammatische und randgrammatische Phänomene (\emph{Ausnahmen}).
%\end{quotation}
%Das ist der Text nach der \texttt{quotation}-Umgebung. Diese Datei ist dazu gemacht, dass alle Workshopteilnehmer einige Vorzüge von \LaTeX\ austesten können.
%
%
%%%%%%%%%%%%%%%%%%%%%%%%%
%\subsubsection{Listen}
%
%Diese Datei ist dazu gemacht, dass alle Workshopteilnehmer einige Vorzüge von \LaTeX\ austesten können.
% 
%\begin{itemize}
%	\item Diese 
%	\item Datei 
%	\item[+] ist 
%	\item dazu 
%	\item gemacht
%\end{itemize} 
%
%Diese Datei ist dazu gemacht, dass alle Workshopteilnehmer einige Vorzüge von \LaTeX\ austesten können.
%
%\begin{enumerate}
%	\item Diese 
%	\item Datei 
%	\begin{enumerate}
%		\item[+] ist 
%		\item dazu 
%		\item gemacht
%	\end{enumerate}
%	\item einige 
%	\item Vorzüge 
%	\item von 
%	\item \LaTeX\ 
%	\item auszutesten
%\end{enumerate}
%
%Diese Datei ist dazu gemacht, dass alle Workshopteilnehmer einige Vorzüge von \LaTeX\ austesten können.
%
%\begin{description}
%	\item[Linguistik:] eine wissenschaftliche Disziplin
%	\begin{itemize}
%		\item Ihr Untersuchungsobjekt ist die Sprache.
%		\item Sie interagiert mit anderen Disziplinen:
%		\begin{enumerate}
%			\item Philosophie
%			\item Psychologie
%			\item Soziologie
%		\end{enumerate}
%	\end{itemize}
%\end{description}


%%%%%%%%%%%%%%%%%%%%%%%%
\section{Hausaufgabe 3}


%%%%%%%%%%%%%%%%%%%%%%%%
\section{Hausaufgabe 4}




%\blindtext

\end{document}

%%%%%%%%END DOCUMENT%%%%%%%%%%%%%
